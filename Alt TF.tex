\documentclass{article}
\usepackage[margin={1.5cm, 1.5cm}]{geometry}

\usepackage{listings}
\usepackage{amsmath}
\usepackage{amssymb}
\usepackage{mathtools}
\usepackage{listings}
\usepackage{color}
\usepackage{tabularx}
\usepackage{pdfrender}
\usepackage{bussproofs}
\usepackage[dvipdfmx]{graphicx}
\usepackage{multicol}
\usepackage{hyperref}
\pagestyle{empty}

\setlength{\columnsep}{1cm}

\title{AltTF: Alternative lock-free TF tree}
\author{荻原湧志, 川島英之}

\begin{document}
\begin{multicols}{2}

\maketitle

\section*{概要}
TFはROSでよく使われるパッケージであり、これは有向木で座標変換を管理する。この木構造にはGiant lockによるパフォーマンス低下の問題、線形補間による一貫性の欠落の問題があった。そこで、我々はトランザクション技術における2PLを導入することにより、この問題を解決した。

\section{序論}


ROSはロボットソフトウェア用のミドルウェアソフトプラットフォームであり、近年多くの研究用ロボットで使われている

グローバル座標系の中でのロボットの位置、ロボットのローカル座標系の中でのセンサーの位置、センサーのローカル座標系の中での物体の位置、という風に別々に管理している

この時、
グローバル座標系での物体の位置を計算するにはグローバル座標系からロボットの位置の計算 * センサーの位置 * 物体の位置を計算すれば計算できる

TFではこのような座標系同士の変換を次の図のように有向森構造で管理できる。座標系の原点をframeと呼び、このフレームをnode、座標系同士の位置関係を並行移動成分・回転成分でedgeで表した木構造で管理する
木構造で管理することにより、先程のようなグローバル座標系での物体Aの位置は、二つのframe間のパスを辿ることで計算できる
ロボットの座標変換の情報は全てこのTFモジュールで管理される

TFモジュールはROSの多くのパッケージで使われているが、これには以下のような問題点がある

Giant lockにより、木構造が大きくなり多くのスレッドがアクセスするケースでは遅くなる

(setTransformのnon atomicityは実はまだ問題にはならない)

提供されているインターフェイスでは「最新」のデータを取るというメソッドでも実際には該当するパスのデータが完全に準備された状態の時間のデータを返している。

また暗黙的に線形補完が行われるため、データの一貫性がなくなる可能性がある





\subsection{貢献}
そこで、データベースのトランザクション機構における並行性制御のプロトコルの一種である2PLを導入した。

その結果、224コアのマシンではこの程度の性能差が出た

また、データ一貫性を保証できるようなsetTransforms, lookupLatestTransformというインターフェースを導入した。

\subsection{論文構成}
本論文の構成は次の通りである。2章では既存のTF森の構造とその問題点について述べる。3章では提案手法であるTF森の再粒度ロックの導入とデータ一貫性のためのインターフェイスの提供について述べる。4章では提案手法の評価結果を述べる。5章では本研究の結論を述べる。6章では今後の課題について述べる。


\section{既存のTF森の構造とその問題点}

TF森の実態はtf2パッケージ中にあるBufferCoreクラス\cite{4} である。
\subsection{構造}
まずはTF森とタイムテーブルから?

各座標系同士の回転移動、並行移動で表現できる位置関係はTF森で表現される

例えば図のようなマップ座標系、ロボット座標系、カメラ座標系、物体の座標系は図のような木構造に対応する。

また、図のフレームA, Bのように他の木とは分離された木が存在してもよい。このため、TF森構造となる。


さて、座標変換の情報は変わらない場合と時刻とともに変わる場合がある

さて、座標変換の情報は主にセンサーからデータが送られてくるたびに計算され、TF森に登録される。このため各フレーム間の座標変換の情報はセンサーの更新頻度に依存し、それぞれ異なるタイミングで更新される。
例えば、マップ座標系からロボット座標系の座標変換を計算するプログラムはLiDARのデータ更新頻度に合わせて自己位置を計算し、マップ座標系からロボット座標系の座標変換を登録する。
カメラ座標系から物体の座標系の座標変換を計算するプログラムはビデオカメラのデータ更新頻度に合わせて物体の位置を計算し、カメラ座標系から物体の座標系の座標変換を登録する。
カメラの更新頻度とLiDARの更新頻度が異なる場合、マップ座標系からロボット座標系の座標変換が登録されるタイミング、カメラ座標系から物体の座標系の座標変換が登録されるタイミングにズレが生じる。マップ座標系から物体の座標系を計算する際にデータ同期の問題が生じてしまう。
これに対処するため、まずTF森は各フレーム間の座標変換情報を過去一定期間保存する。これは図のように表現できる

% map -> robot    
% robot -> camera
% camera -> apple

図は各フレーム間の座標変換情報が提供される時刻を表す。横軸が時間軸を表し、左側が過去、右側が最新の時刻を表す。

灰色線は対応する時刻の座標変換情報が登録されたことを表す。
図の点線の位置でのmapからappleへの座標変換を計算する際、各フレーム間の座標変換情報が必要になる。robotからcameraへの座標変換情報は常にある物として扱われるが、mapからrobot、cameraからappleの座標変換データは該当する時刻では登録されていない。ここで、TFは前後のデータから線形補間を行うことにより座標変換データを計算する。




文字列であるframeはTF森内部では整数型のidで管理している
frameからframe idを検索するテーブル、及びframe idからframeを検索するテーブルが存在する

座標変換はTransformStorageで表現され、これは以下で構成される
回転成分を表すrotation。これはQuaternionで表現される
並行移動成分を表すtranslation。これはVector3で表現される
座標変換の時刻を表すstamp。これはros::Time型で表現される
座標変換の親フレームのframe idを表すframe\_id。これは整数型で表現される
座標変換の子フレームのframe idを表すchild\_frame\_id。これは整数型で表現される

TimeCacheはある子フレームにおける親フレームへの座標変換を時系列的に管理する。これはTransformStorageのdequeで構成され、ユーザーが座標変換情報を登録する際には先頭にpushし、保存していた座標変換の時刻が10秒以上過去のものとなればdequeから追い出す。

StaticCacheは先程の例のrobot->sensor間の座標変換のように不変なある子フレームにおける親フレームへの座標変換情報を管理する。

TimeCacheとStaticCacheはTF森における子フレームから親フレームへのエッジ及び子フレームのノードを表す。TimeCacheInterfaceはTimeCacheとStaticCacheの親クラスであり、各フレームidに対応するTimeCacheInterfaceのポインタを管理する配列が各フレームに対応するTimeCacheとStaticCacheを管理する。


\subsection{検索}

TFにはlookupTransformというメソッドがあり、


\subsection{更新、挿入}





\subsection{問題点}

\section{提案手法}

提案手法とsnapshot latestの概念の導入について
時系列データにはいかなる変更も加えられないためserializabilityを導入する必要はない
また、最新のデータを見るにはserializabilityが必要

\section{評価}
実験を行う

\section{結論}

データの正確性は必須

\section{今後の課題}

ここでは取り上げなかったTF木の問題点として、部分的に

	
\begin{thebibliography}{99}

\bibitem{1} Simon J. Julier and Jeffrey K. Uhlmann "Building a million beacon map", Proc. SPIE 4571, Sensor Fusion and Decentralized Control in Robotic Systems IV, (4 October 2001)

\bibitem{2} M. Quigley, K. Conley, B. P. Gerkey, J. Faust, T. Foote, J. Leibs, R. Wheeler, and A. Y. Ng, “Ros: an open-source robot operating system,” in ICRA Workshop on Open Source Software, 2009.

\bibitem{3} T. Foote, "tf: The transform library," 2013 IEEE Conference on Technologies for Practical Robot Applications (TePRA), 2013, pp. 1-6, doi: 10.1109/TePRA.2013.6556373.

\bibitem{4} "BufferCore.h", https://github.com/ros/geometry2/blob/noetic-devel/tf2/include/tf2/buffer\_core.h

\end{thebibliography}
	
\end{multicols}	
\end{document}
