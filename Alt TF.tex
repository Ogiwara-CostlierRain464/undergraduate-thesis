\documentclass{article}
\usepackage[margin={1.5cm, 1.5cm}]{geometry}

\usepackage{listings}
\usepackage{amsmath}
\usepackage{amssymb}
\usepackage{mathtools}
\usepackage{listings}
\usepackage{color}
\usepackage{tabularx}
\usepackage{pdfrender}
\usepackage{bussproofs}
\usepackage[dvipdfmx]{graphicx}
\usepackage{multicol}
\usepackage{hyperref}
\pagestyle{empty}

\setlength{\columnsep}{1cm}

\title{AltTF: Alternative lock-free TF tree}
\author{荻原湧志}

\begin{document}
\begin{multicols}{2}

\maketitle

\section*{概要}
TFはROSでよく使われるパッケージであり、これは有向木で座標変換を管理する。この木構造にはGiant lockによるパフォーマンス低下の問題、線形補間による一貫性の欠落の問題があった。そこで、我々はトランザクション技術における2PLを導入することにより、この問題を解決した。

\section{序論}


ROSはロボットソフトウェア用のミドルウェアソフトプラットフォームであり、近年多くの研究用ロボットで使われている

グローバル座標系の中でのロボットの位置、ロボットのローカル座標系の中でのセンサーの位置、センサーのローカル座標系の中での物体の位置、という風に別々に管理している

この時、
グローバル座標系での物体の位置を計算するにはグローバル座標系からロボットの位置の計算 * センサーの位置 * 物体の位置を計算すれば計算できる

TFではこのような座標系同士の変換を次の図のように有向森構造で管理できる。座標系の原点をframeと呼び、このフレームをnode、座標系同士の位置関係を並行移動成分・回転成分でedgeで表した木構造で管理する
木構造で管理することにより、先程のようなグローバル座標系での物体Aの位置は、二つのframe間のパスを辿ることで計算できる
ロボットの座標変換の情報は全てこのTFモジュールで管理される

TFモジュールはROSの多くのパッケージで使われているが、これには以下のような問題点がある

Giant lockにより、木構造が大きくなり多くのスレッドがアクセスするケースでは遅くなる

(setTransformのnon atomicityは実はまだ問題にはならない)

提供されているインターフェイスでは「最新」のデータを取るというメソッドでも実際には該当するパスのデータが完全に準備された状態の時間のデータを返している。

また暗黙的に線形補完が行われるため、データの一貫性がなくなる可能性がある





\subsection{貢献}
そこで、データベースのトランザクション機構における並行性制御のプロトコルの一種である2PLを導入した。

その結果、224コアのマシンではこの程度の性能差が出た

また、データ一貫性を保証できるようなsetTransforms, lookupLatestTransformというインターフェースを導入した。

\subsection{論文構成}
本論文の構成は次の通りである。2章では既存のTF森の構造とその問題点について述べる。3章では提案手法であるTF森の再粒度ロックの導入とデータ一貫性のためのインターフェイスの提供について述べる。4章では提案手法の評価結果を述べる。5章では本研究の結論を述べる。6章では今後の課題について述べる。


\section{既存のTF森の構造とその問題点}

まずはTF2

既存のTFの構造と問題点について

\section{提案手法}

提案手法とsnapshot latestの概念の導入について
時系列データにはいかなる変更も加えられないためserializabilityを導入する必要はない
また、最新のデータを見るにはserializabilityが必要

\section{評価}
実験を行う

\section{結論}

データの正確性は必須

\section{今後の課題}

ここでは取り上げなかったTF木の問題点として、部分的に

	
\begin{thebibliography}{99}

\bibitem{1} Simon J. Julier and Jeffrey K. Uhlmann "Building a million beacon map", Proc. SPIE 4571, Sensor Fusion and Decentralized Control in Robotic Systems IV, (4 October 2001)


\end{thebibliography}
	
\end{multicols}	
\end{document}
