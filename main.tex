\documentclass[a4paper]{jreport}	% 日本語の場合

\usepackage{masterThesisJa-Ja}
\usepackage[dvipdfmx]{graphicx}
\usepackage{hyperref}
\usepackage{breakurl}
\pagestyle{empty}
\usepackage{newtxtext,newtxmath}
\usepackage{algpseudocode}
\usepackage{algorithm}
\usepackage{array}
\setcounter{tocdepth}{3}
\setcounter{page}{-1}


%TODO: 学会ならもっとplotは疎にする、あるいは綺麗なデータをとる

% 【必須】主題:\maintatile{日本語}{英語}
\maintitle{並行性制御法によるROS TFの高品質化}{Make ROS TF high quality in concurrency control method}

% 【任意】副題:\subtitle{日本語}{英語}
% 副題が不要な場合は次の行をコメントアウトしてください
%\subtitle{}{}

% 【必須】発表年月:\publish{年}{月}
\publish{2022}{1}

% 【必須】学生情報:\student{学籍番号/CNSアカウント}{氏名(日本語:氏名の間は1文字空ける)}{氏名(英語:Twins登録の表記)}
\student{71970013 / t19501yo}{荻原 湧志}{Yushi Ogiwara}

% 【必須】概要:\abst{概要}
\abst{
Robot Operating System(ROS) はロボットソフトウェア用のミドルウェアソフトプラットフォーム であり、近年多くの研究用ロボットで用いられている。TF ライブラリは ROS で頻繁に使用されるパッケージであり、各座標系間の変換を有向木構造として管理し、効率的な座標変換情報の登録、座標変換の計算を可能にした。
この有向木構造には非効率な並行性制御によりアクセスが完全に逐次化され、アクセスするスレッドが増えるに従ってパフォーマンスが低下する問題、及び座標変換の計算時にその仕様によって最新のデータを参照しないという問題があることがわかった。そこで、我々はデータベースの並行性制御法における 細粒度ロッキング法、及び 2PL を応用することにより、これら問題を解決した。提案手法では既存手法と比べ最大257倍のスループット、最大282倍高速化したレイテンシ、最大132倍のデータ鮮度となることを示した。 

}

% 【必須】研究指導教員(氏名の間は1文字空ける):\advisors{主研究指導教員}{副研究指導教員}
%\advisors{川島 英之}{}
\advisors{川島 英之}


% 以下,本文を出力
\begin{document}

\makecover

\addtolength{\textheight}{-5mm}	% 本文の下限を5mm上昇
\setlength{\footskip}{15mm}	% フッタの高さを15mmに設定
\fontsize{11pt}{15pt}\selectfont

% 目次・表目次を出力
\pagebreak\setcounter{page}{1}
\pagenumbering{roman} % I, II, III, IV
\pagestyle{plain}
\tableofcontents
\listoffigures

% 本文
\parindent=1zw	% インデントを1文字分に設定
\pagebreak\setcounter{page}{1}
\pagenumbering{arabic} % 1,2,3
\pagestyle{plain}

% 章:\chapter{}
% 節:\section{}
% 項:\subsection{}


\chapter{はじめに}
% プレースホルダ
\section{研究背景}

2022年01月14日にREPORTOCEANが発行した新しいレポートによると、自律移動型ロボットの世界市場は予測期間2021-2027年にかけて19.6%以上の健全な成長率で成長すると予測されている~\cite{report1}。操縦者を必要とせず、自律的な移動が可能な自律移動型ロボットは、作業負担の大きい業務の代替や、労働力不足を解消する手段の一つとして関心が高まっている。小売・卸売倉庫ではインターネットショッピングによる受発注作業の継続的な増加と人手不足が深刻化している。企業は、こうした課題解決のために、商品棚やパレットを運ぶ自律移動型ロボットの導入を加速させており、さらに商品管理から出荷前の棚出し梱包までを自動化することで、作業員の作業量削減と能力に依存しないオペレーションの構築を目指している。このような自動化の実現に向けて、自律移動型ロボットが積極的に活用されていくとされる~\cite{report2}。

ロボットを使って作業を行う場合、ロボット自身がどこにいるのか、ロボットにはどこにどんなセンサーがついており、また周りの環境のどこにどんなものがあるかをシステムが把握することが重要である~\cite{tf}。例えば、図\ref{fig:room} のように部屋の中にロボットと、ロボットから観測できる二つの物体があるケースを考える。図中にてロボットは円形、物体は星形で表現され、ロボットが向いている方向は円の中心から円の弧へつながる直線の方向で表される。直角に交わる二つの矢印は座標系を表し、交点が座標系の原点、二つの矢印が反時計回りにそれぞれX・Y軸を表す。

座標系とは空間中の物体の座標を定める方法を与えるものであり、座標系内の物体は原点からのX・Y軸方向の距離の組で座標が与えられる。空間全体の座標系をグローバル座標系と呼び、その中にある個別の物体それぞれにローカル座標系を設定することによって、全体空間の中でのそれぞれのオブジェクトの変化を扱いやすくする。ここでは地図、ロボット、オブジェクトそれぞれに座標系が与えられ、グローバル座標系として地図座標系、ローカル座標系としてロボット座標系・オブジェクト座標系として扱う。地図座標系の原点は地図の左下端に設定され、ロボット座標系・オブジェクト座標系はそれぞれロボット・オブジェクトの中心点と向きから決められる。

システムはロボットに搭載されたセンサーからのデータを元に各座標系間の位置関係を随時更新し、この位置関係は三次元ベクトルで表現される平行移動成分と、四元数で表現される回転成分で表現できる。例えば、自己位置推定プログラムはLiDARから点群データが送られてくるたびにそれを地図データと比較して自己位置を計算し、ロボットが地図座標系にてどの座標に位置するか、ロボットがどの方向を向いているかといった、地図座標系からロボット座標系への位置関係を更新する~\cite{tf}。物体認識プログラムはカメラからの画像データが送られてくるたびに画像中の物体の位置を計算し、ロボット座標系から物体座標系への位置関係を更新する。これを表\ref{table:sensor-prog}にまとめた。

\begin{table}[h]
	\centering
	 \begin{tabular}{clll}
   \hline
    座標変換 & 座標変換を計算するプログラム & 座標変換に使われるデータ \\
    \hline \hline
   地図座標系からロボット座標系 & 自己位置推定プログラム & LiDARからの点群データ \\
   ロボット座標系から物体座標系 & 物体認識プログラム & カメラからの画像データ \\            
   \hline
  \end{tabular} 
  \caption{座標系間の位置関係を更新するセンサーとプログラム}
	\label{table:sensor-prog}
\end{table}


このように、各座標系間の位置関係の更新にはそれぞれ異なるセンサー、プログラムが使われる。各センサーの計測周期、及び各プログラムの制御周期は異なるため、各座標系間の位置関係の更新頻度も異なるものとなる。図\ref{fig:sensor-sync}では地図座標系からロボット座標系への位置関係データと、ロボット座標系から物体座標系への位置関係データのそれぞれが登録されたタイミングを黒いセルで表し、それぞれが異なるタイミングで登録されることを示している。

\begin{figure}[h] 
\centering{\includegraphics[width=12cm]{room}}	
\caption{部屋の中のロボット}
\label{fig:room}
\end{figure}

\begin{figure}[h] 
\centering{\includegraphics[width=12cm]{sensor-sync}}	
\caption{位置関係の登録のタイムライン}
\label{fig:sensor-sync}
\end{figure}


ここで、地図中での物体の位置を把握するために、地図座標系から物体座標系への位置関係を取得する方法について考える。三次元ベクトル$(x, y, z)^{\mathrm{T}}$を同次座標で表した $\mathbf{p} = (x, y, z, 1)^{\mathrm{T}}$と、四元数$\mathbf{q}$で表現される回転がある時、$\mathbf{p}$を$\mathbf{q}$で回転して得られる同次座標$\mathbf{p'}$は
\begin{equation}
	\mathbf{p'} = \mathbf{qp}(\mathbf{q})^{-1}
\end{equation}

で与えられる。これにより、地図座標系からロボット座標系への位置関係が平行移動成分が$\mathbf{p}_{\mathrm{map \rightarrow robot}}$、回転成分が$\mathbf{q}_{\mathrm{map \rightarrow  robot}}$、ロボット座標系から物体座標系への位置関係が平行移動成分が$\mathbf{p}_{\mathrm{robot \rightarrow object}}$、回転成分が$\mathbf{q}_{\mathrm{robot \rightarrow object}}$で与えられる時、地図座標系から物体座標系への平行移動成分$\mathbf{p}_{\mathrm{map \rightarrow object}}$、回転成分$\mathbf{q}_{\mathrm{map \rightarrow object}}$はそれぞれ

\begin{equation}
\begin{aligned}[t]
	\mathbf{p}_{\mathrm{map \rightarrow object}} &= 
	\mathbf{q}_{\mathrm{robot \rightarrow object}}
	\mathbf{p}_{\mathrm{map \rightarrow robot}}
	(\mathbf{q}_{\mathrm{robot \rightarrow object}})^{-1} \\
	\mathbf{q}_{\mathrm{map \rightarrow object}} &= \mathbf{q}_{\mathrm{robot \rightarrow object}} 
	\mathbf{q}_{\mathrm{map \rightarrow  robot}}
\end{aligned}
\label{equation:cross}
\end{equation}

で与えられる。このように、二つの座標系間の位置関係はその座標系どうしを繋ぐような座標変換を平行移動成分、回転成分それぞれ掛け合わせることによって計算できる。

ロボット座標系から地図座標系への位置関係(平行移動成分$\mathbf{p}_{\mathrm{robot \rightarrow map}}$と回転成分$\mathbf{q}_{\mathrm{robot \rightarrow map}}$)を得るには、地図座標系からロボット座標系への位置関係の逆変換を取れば良い。これは、

\begin{equation}
\begin{aligned}[t]
	\mathbf{q}_{\mathrm{robot \rightarrow map}} &= {\mathbf{q}_{\mathrm{map \rightarrow robot}}}^{-1}
	\\
	\mathbf{p}_{\mathrm{robot \rightarrow map}} &= \mathbf{q}_{\mathrm{robot \rightarrow map}}(-\mathbf{p}_{\mathrm{map \rightarrow robot}})
	(\mathbf{q}_{\mathrm{robot \rightarrow map}})^{-1}
\end{aligned}
\label{equation:inv}
\end{equation}

で与えられる。


図\ref{fig:sensor-sync}のように各変換データは異なるタイミングで来るため、最新の変換データを取得するプログラムは次のように複雑なものとなる。

Aの時刻で地図座標系から物体座標系への変換データを計算しようとするとロボット座標系から物体座標系への最新の変換データ$\alpha$を取得できるが、地図座標系からロボット座標系への変換データは時刻Aにはまだ存在しない。このため、最新の変換データ$\theta$を使う、もしくは過去のデータを元にデータの補外をし、時刻A時点での変換データを予測するという選択肢が挙げられる。

Bの時刻で地図座標系から物体座標系への変換データを計算しようとすると地図座標系からロボット座標系への変換データ$\theta$を取得できるが、ロボット座標系から物体座標系への変換データはその時間には提供されていない。このため、$\alpha$と$\beta$のデータから線形補間を行う、もしくは$\beta$か$\alpha$のデータを利用するという選択肢が挙げられる。

各変換データが異なるタイミングで来るために選択肢が複数生じるという問題以外にも、情報の分散という問題もある。地図座標系からロボット座標系への位置関係、ロボット座標系から物体座標系への位置関係はそれぞれ表\ref{table:sensor-prog}のように別々のプログラムで計算されているため、座標系同士の位置関係に関する情報は別々のプログラムで管理され、分散した状態となっている。

このように、ROSの開発初期には複数の座標変換の管理が開発者共通の悩みの種であると認識されていた。このタスクは複雑なために、開発者がデータに不適切な変換を適用した場合にバグが発生しやすい場所となっていた。また、この問題は異なる座標系同士の変換に関する情報が分散していることが多いことが課題となっていた~\cite{tf}。

\section{TFライブラリ}
\label{section:intro-tf}

この問題を解決するためにTFライブラリが提案された。TFライブラリはROS上で動作し、各座標系間の変換を有向木構造として一元管理し、効率的な座標系間の変換情報の登録、座標系間の変換の計算を可能にした~\cite{tf}。図\ref{fig:room}を表す木構造は図\ref{fig:room-tree}で表現できる。ノードが各座標系を表し、エッジは子ノードから親ノードへの変換データが存在することを表す。

\begin{figure}[h] 
\centering
\includegraphics[width=5cm]{tree}	
\caption{図\ref{fig:room}に対応する木構造}
\label{fig:room-tree}
\end{figure}

ノードはTFではフレームと呼ばれ、フレーム中の文字列は各座標系に対応するフレーム名である。図\ref{fig:room-tree}では地図座標系のフレーム名はmap、ロボット座標系のフレーム名はrobot、物体1の座標系のフレーム名はobject1となる。親フレームへ張られたエッジは子フレームから親フレームへポインタが貼られていることを表し、子フレームから親フレームを辿ることができる。しかしながら、親フレームから子フレームを辿ることはできない。このため、mapからobject1への座標変換を計算するにはobject1からmapへの座標変換の計算をし、その逆変換を取る必要がある。

上述したように、各フレーム間の座標変換情報はそれぞれ異なるタイミングで登録される。これに対処するため、TFでは各フレーム間の座標変換情報を10秒間保存する。図\ref{fig:room-tree}において各フレーム間の座標変換情報が登録されたタイミングを表すのが図\ref{fig:room-timeline}である。横軸は時間軸を表し、左側が過去、右側が最新の時刻を表す。黒色のセルはデータがその時刻に登録されたことを表す。時刻Aではrobotからmapへの座標変換の情報が得られるが、object1からrobotへの座標変換の情報は時刻Aには存在しない。そこで、TFでは前後のデータから線形補間を行うことにより該当する時刻の座標変換データを計算する。つまり、TFはある時刻の座標変換データが保存されているか線形補間で取得できる時に、その時刻の座標変換データを提供できる、とみなす。灰色の領域は線形補間により座標変換データが提供可能な時間領域を表す。

\begin{figure}[h] 
\centering
\includegraphics[width=15cm]{room-timeline}	
\caption{図\ref{fig:room-tree}における位置関係登録のタイムライン}
\label{fig:room-timeline}
\end{figure}

図\ref{fig:room-tree}における位置関係登録のタイムラインが図\ref{fig:room-timeline}のようになっているとき、TFではobject1からmapへの最新の位置関係は次のように計算する。

まず、object1からmapへの各エッジを確認する。ここではobject1からmapへの各エッジはobject1$\rightarrow$robot, robot$\rightarrow$mapであることがわかる。

次に、どのエッジにおいてもなるべく最新の座標変換を提供できる時刻を確認する。図\ref{fig:room-timeline}を確認すると、object1$\rightarrow$robot、robot$\rightarrow$mapにおいて最新の座標変換情報が登録された時刻が最も古いのはrobot$\rightarrow$mapである。このため、時刻Aがここでは要件を満たす。

最後に、時刻Aでの各エッジのデータを取得し、それらを掛け合わせる。robot$\rightarrow$mapについては登録されたデータを使い、object1$\rightarrow$robotについては線形補間によってデータを取得する。

\section{研究課題}
前述したようにTFはロボットシステム内部の座標系間の位置関係を一元管理する。しかしながら、これには以下のような問題点が挙げられる。

\subsection*{問題1:ジャイアント・ロック}
TFの木構造には複数のスレッドが同時にアクセスするため並行性制御が必要となるが、既存のTFでは一つのスレッドが木構造にアクセスしている際は他のスレッドは木構造にアクセスできないアルゴリズムとなっている。複数スレッドが木構造の別々の部分にアクセスするケース、及び複数スレッドが木構造の同じ部分のアクセスしているが全て読み込みアクセスのケースなど、排他制御が必要ではないケースにおいてもアクセスが完全に逐次化されている。これにより、マルチコアが常識となっている現代ではスループットやレイテンシに問題が生じる可能性がある。

\subsection*{問題2a:データの鮮度}
上述のように、TFのフレーム間の座標変換計算インターフェースはある時刻の座標変換データが保存されているか線形補間で取得できる時に、その時刻の座標変換データを提供できるという仕様のため、最新のデータを使わない可能性がある。同時刻のデータを元に座標変換を行うためデータの時刻同期性はあるが、最新の座標変換データを使わないためデータの鮮度は失われる。これにより、ロボットの制御や自己位置推定に問題が生じる可能性がある。現在、TFライブラリには最新の座標変換データをもとにフレーム間の座標変換計算をするインターフェイスは無い。

また、この仕様により座標系間の位置関係があまり変わらない場合についても頻繁にデータを登録する必要があり、無駄な処理が発生する。

\subsection*{問題2b:データの一貫性}
問題2aの解決策として、最新の座標変換データをもとにフレーム間の座標変換計算をするインターフェイスを提供するだけでは不十分である。これは、複数の座標変換データを登録している途中に最新の座標変換データをもとにフレーム間の座標変換計算をすると、ユーザーが期待するデータの一貫性がなくなる可能性があるからである。

%くわしい話をここでもすべきだろうか?
\section{貢献}
問題1については、並行性制御法における細粒度ロッキング法を適用することによって解決した。細粒度ロッキング法は、並行性制御においてロックするデータの単位をなるべく小さくし、並行性を向上させる手法である。これにより、既存手法ではジャイアントロックによってTFへのアクセスは完全に逐次化されていたが、提案手法では細粒度ロックによってフレーム単位でのロックとなり、複数のフレームに並行にアクセスが可能になった。細粒度ロックを実装した場合のスループットは最大で11,574,200tps、レイテンシは高々0.7msとなった。また既存手法と比べ細粒度ロックを実装した場合はスループットは最大243倍、レイテンシは最大172倍高速化した。

問題2a、2bについては、データベースの並行性制御法における2PLを適用することにより、複数の座標変換の最新のデータをatomicに取得するインターフェース(lookupLatestTransformXact)、及び複数の座標変換の最新のデータをatomicに更新するインターフェース(setTransformsXact)を提供することによって解決した。2PLとは、複数のデータに対するロック・アンロックのタイミングを二つのフェーズに分けることにより並行性を向上させつつ、複数のデータ操作をatomicに行えるようにする手法である。これにより、複数のデータの最新の座標変換情報の読み込み・書き込みを効率的にatomicに行えるようになった。既存手法で提供されているインターフェイスではジャイアントロックによりTFへのアクセスが逐次化され、また§\ref{section:intro-tf}にて説明したように時刻の同期をとるという仕様のために過去のデータを参照しデータの鮮度が落ちるという問題があったが、lookupLatestTransformXactとsetTransformsXactを使うことによりこれは解決できた。既存手法と比べ、lookupLatestTransformXactとsetTransformsXactを使った場合にはデータの鮮度は最大で132倍となった。また、既存手法と比べ最大257倍のスループット、最大282倍高速化したレイテンシとなり、このインターフェイスはスループットとレイテンシにおいても既存手法より優れていることを示した。既存のTFライブラリ~\cite{ros-geometry2}に1236行分の変更を加え、提案手法を実装したコードを~\cite{ogiwara-geometry2}に公開した。

\section{構成}
本論文の構成は次の通りである。第二章では関連研究について述べる。第三章では既存のTFの木構造とその問題点について述べる。第四章では提案手法である木構造への再粒度ロックの導入とデータの鮮度、データの一貫性のためのインターフェイスの提供について述べる。第五章では提案手法の評価結果を述べる。第六章では本研究の結論と今後の課題について述べる。

\chapter{関連研究}

\section{データベース分野におけるロボット研究}
%
筆者の知る限り、データベース分野ではロボットを対象にした研究論文はトップ会議、トップジャーナルでは発表されたことがない。
%
他方、産業界における珍しい例としてはGAIA platform~\cite{gaia}が挙げられる。GAIA platformはリレーショナルデータベースと、そのデータベースに変更が加えられた時の処理をC++で宣言的に記述できる仕組みを組み合わせることにより、イベントドリブンなフレームワークでロボットや自動運転システムを構築するものである。
このデータベースアクセスはトランザクションを用いて実行される。GAIAチームにはsnapshot isolation提案者も含まれており、そのトランザクションアーキテクチャには一定の頑健性があることが期待される。

\section{ロボット分野におけるデータベース研究}
TFライブラリのようにデータを時系列的に管理するライブラリとしてSSM~\cite{ssm}が挙げられる。SSMでは各種センサデータを共有メモリ上のリングバッファで管理することにより、時刻の同期を取れたデータを高速に取得することができる。
% 移動ロボット用センサ情報処理ミドルウェアの開発 か?
ROSはロボットソフトウェア用のミドルウェアソフトプラットフォームであり、近年多くの研究用ロボットで用いられている。産業用途にも利用可能にするためにROSの次世代バージョンであるROS2~\cite{ros2}の開発が進んでいるが、並行性制御アルゴリズムはROSから変わっておらず、本研究のようなアプローチはない。
\section{トランザクション研究}
データベース分野における高速並行性研究におけるトランザクション処理システムとして2PL~\cite{2PL}、Silo~\cite{silo}、MOCC~\cite{MOCC}、Cicada~\cite{Cicada}が挙げられる。古典的なトランザクション処理システムはハードウェアのコア数が少なく、メモリが小さい中でいかに効率的に処理を行うかに重きが置かれてきたため、2PLなどの悲観的ロッキング法が主に使われていた。
しかしながら、近年のハードウェアはメニーコア化と大容量メモリが前提のシステムとなっており、従来手法では必ずしも性能が最大限出るとは言えない。そこでSilo~\cite{silo}やCicada~\cite{Cicada}などの楽観的並行性制御法が提案された。これらは近年のハードウェアを前提に設計されているため、それまでのシステムと比べ劇的に高い性能を実現する。
具体的な理由はcontetion pointの消失である。Multi-version concurrency control (MVCC)の場合、single shared counter を用いるため、メニーコア環境では性能が劇的に劣化する。そのため、snapshot isolationやmulti-version timestamp orderingのようなscheduling spaceの広大な手法であっても、高い性能を出すことは難しい。メニーコア環境ではMVCCよりも2PLの方が高い性能を有することが知られている~\cite{ccbench}.

\chapter{既存のTFの木構造とその問題点}
\section{TFライブラリの使用例}

\begin{figure}[h] 
\centering
\includegraphics[width=8cm]{aqua.jpeg}	
\caption{ロボット「Aqua」}
\label{fig:aqua}
\end{figure}


ロボット「Aqua」は筑波大学知能ロボット研究室で開発された自律移動ロボットであり、前輪2輪が駆動輪、後輪のキャスタによる従動輪で構成される差動2輪駆動方式のロボットである~\cite{Aqua}。制御にはT-frog ProjectのモータドライバTF-2MD3-R6を使用し、本ロボットの移動制御を行う。計算機部分についてはCPUはIntel Core i7-9700K Processor、GPUはGeForce GTX 1050 Ti、RAMは32GBで構成されている。機体の上部、旋回中心に自己位置推定に使用するVelodyne社製のVLP-16が1台、前後方の障害物の検出に使用する北陽電機社製のUTM--30LXを機体下部前方と後方に2台搭載している。また、BUFFALO社製のWebカメラBSK200MBKが3台、PointGray社製のGrasshopper3カメラGS3-U3-123S6M-Cを用いて構築したステレオカメラと北陽電機社製YVT--X002をそれぞれ1台搭載している。Aquaの外観を図\ref{fig:aqua}に示す。

\begin{figure}[h] 
\centering
\includegraphics[width=8cm]{aqua-tf.jpeg}	
\caption{AquaにおけるTFの木構造}
\label{fig:aqua-tf}
\end{figure}

Aquaのシステムではロボットの自己位置とセンサーの位置を管理するためにTFライブラリを用いている。AquaにおけるTFの木構造を図\ref{fig:aqua-tf}~\cite{Aqua}に示す。エッジの横に書かれているのはフレーム間の座標変換を更新するプログラム名である。  Aquaでは、VLP-16からの三次元点群データを元に、Autoware~\cite{autoware}のndt\_matchingモジュール~\cite{ndt_matching}を用いて自己位置推定を行う。ndt\_matchingによって与えられる自己位置推定は三次元空間での自己位置のため、Aquaではその三次元空間での自己位置を二次元空間での自己位置に変換するpose\_filterパッケージを自作して、ロボットの自己位置をTFライブラリに登録する。

このように、TFライブラリは自律移動ロボットでは広く利用されている。

\section{構造}
TFライブラリでは図\ref{fig:sample-tree}のように各座標系間の位置関係を木構造で管理する。ノードが各座標系を表し、エッジは子ノードから親ノードへの座標変換データが存在し、また親ノードへポインタが貼られていることを表す。このため子ノードから親ノードへ辿ることはできるが、親ノードから子ノードを辿ることはできない。ノードはフレームと呼ばれ、フレーム内の文字列は各座標系に対応するフレーム名である。

各フレーム間の座標変換情報は、フレーム間のエッジに10秒間保存される。このため、各フレーム間の座標変換情報が登録された時刻を図\ref{fig:general-timeline}のようなタイムラインで表現できる。黒のセルは登録されたデータを表し、灰色のセルは線形補間により座標変換データが取得可能な時刻を表す。横軸が時間軸を表し、左側が過去、右側が最新の時刻を表す。

\begin{figure}[h] 
\centering
\includegraphics[width=4cm]{sample-tree}	
\caption{TFの木構造}
\label{fig:sample-tree}
\end{figure}

\begin{figure}[h] 
\centering
\includegraphics[width=15cm]{general-timeline.png}
\caption{タイムライン}
\label{fig:general-timeline}
\end{figure}

\section{TFの木構造のインターフェイス}
\label{section:tf-interface}

TFライブラリの木構造のインターフェイスは~\cite{buffer-core}に公開されており、このうち主に以下の表\ref{table:tf-interface}のインターフェイスが使われている。

\begin{table}[h!]
\centering
\begin{tabular}{ | m{5cm} | m{5cm} | } 
  \hline
  setTransform & フレーム間の座標変換を登録する \\ 
  \hline
  lookupTransform & フレーム間の座標変換を計算する \\ 
  \hline
  canTransform & フレーム間の座標変換が計算できるかを確認する \\ 
  \hline
  allFramesAsYAML & 全てのフレームの情報をYAML形式で取得する \\
  \hline
  allFramesAsString & 全てのフレームの情報を文字列で取得する \\
  \hline
  addTransformableCallback & フレーム間の座標変換が登録された時のコールバック関数を追加する \\
  \hline
  removeTransformableCallback & addTransformableCallbackで追加したコールバック関数を削除する \\
  \hline
\end{tabular}	
\caption{TFの木構造のインターフェイス}
\label{table:tf-interface}
\end{table}

この中でも頻繁に使用されるのが\textit{lookupTransform}と\textit{setTransform}であり、TFの木構造へのアクセスは主に、高頻度で\textit{lookupTransform}のみ呼び出す読み込み専用スレッドと、高頻度で\textit{setTransform}のみ呼び出す書き込み専用スレッドで構成される。
本研究ではこの二つのインターフェイスの改善を行う。


\section{lookupTransform}
\label{section:lookupTrnasform}

二つのフレーム間の座標変換情報を取得するには\textit{lookupTransform}を使う。lookupTransformの実装は~\cite{lookupTransform}に公開されており、登録されたフレームが図\ref{fig:sample-tree}、登録された座標変換情報のタイムラインが図\ref{fig:general-timeline}の状況において、\textit{lookupTransform}を用いてフレームcからフレームdへの座標変換を計算する動作例を説明する。

\begin{enumerate}
	\item フレームcから木構造のルートフレームへのエッジを取得する。ここではルートフレームはaであり、フレームcからフレームaへのエッジはc$\rightarrow$bとb$\rightarrow$aである。
	\item フレームdから木構造のルートフレームへのエッジを取得する。同じようにルートフレームはaとなり、フレームdからフレームaへのエッジはd$\rightarrow$aとなる。
	\item 得られた三つのエッジであるc$\rightarrow$b、b$\rightarrow$a、d$\rightarrow$aから、どのエッジにおいてもなるべく最新の座標変換を提供できる時刻を確認する。ここでは時刻Aが要件を満たす。
	\item 時刻Aにおける各エッジの座標変換データを計算する。b$\rightarrow$aについては登録されたデータを利用でき、c$\rightarrow$bとd$\rightarrow$aについては線形補間でデータを生成する。
	\item フレームcからフレームaへの座標変換と、フレームaからdへの座標変換を掛け合わせる。フレームcからフレームaへの座標変換はc$\rightarrow$bとb$\rightarrow$aの座標変換をかけ合わせれば得られ、フレームaからdへの座標変換はd$\rightarrow$aの逆変換から得られる。
\end{enumerate}
また、\textit{lookupTransform}は指定した時刻のデータを取得することもできる。時刻Bにおけるフレームcからフレームdへの座標変換は次のアルゴリズムで得られる。

\begin{enumerate}
	\item フレームcから木構造のルートフレームへの時刻Bにおける座標変換を取得する。フレームcから木構造のルートフレームへのエッジはc$\rightarrow$b、b$\rightarrow$aとなり、それぞれの座標変換は線形補間によって得られる。
	\item フレームdから木構造のルートフレームへの時刻Bにおける座標変換を取得する。フレームdから木構造のルートフレームへのエッジはd$\rightarrow$aとなり、座標変換は線形補間によって得られる。
	\item フレームcからフレームaへの座標変換と、フレームaからdへの座標変換を掛け合わせる。フレームaからdへの座標変換はd$\rightarrow$aの逆変換から得られる。
\end{enumerate}

\textit{lookupTransform}の疑似コードをアルゴリズム\ref{algo:lookupTransform}で説明する。\textit{lookupTransform}にて最新の座標変換を計算するには\ref{code:getLatest}行目のように、まず\textit{getLatestCommonTime}インターフェイスを呼び出し、sourceフレームからtargetフレームへの最新の座標変換を計算できる時刻を取得する。

\begin{figure}[h] 
\centering
\includegraphics[width=5cm]{same-root}	
\caption{sourceとtargetの祖先がrootのみのパターン}
\label{fig:same-root}
\end{figure}

\begin{figure}[h] 
\centering
\includegraphics[width=5cm]{s-t}	
\caption{sourceの祖先がtargetのパターン}
\label{fig:s-t}
\end{figure}

\begin{figure}[h] 
\centering
\includegraphics[width=5cm]{t-s}	
\caption{targetの祖先がsourceのパターン}
\label{fig:t-s}
\end{figure}


\begin{figure}[h] 
\centering
\includegraphics[width=5cm]{same-parent}
\caption{sourceとtargetの祖先がrootだけではないパターン}
\label{fig:same-parent}
\end{figure}

sourceフレームとtargetフレームの位置関係には以下の四パターンがあり、それぞれ図\ref{fig:same-root}〜図\ref{fig:same-parent}で示す。図中のsはsourceフレームを、tはtargetフレームを表す。

\begin{enumerate}
	\item sourceとtargetの祖先がrootのみのパターン
	\item sourceの祖先がtargetのパターン
	\item targetの祖先がsourceのパターン
	\item sourceとtargetの祖先がrootだけではないパターン
\end{enumerate}

このうち、2〜4のパターンについてはsourceフレームからtargetフレームへの最新の座標変換を計算できる時刻を取得する際に、次に説明するようにsourceからrootへ、もしくはtargetからrootへ辿る処理を省略できる。

パターン2ではsourceからrootへのパスを辿る際に途中でtargetを見つけ、rootまで辿ること無しにその時点で取得できたデータから最新の座標変換を計算できる時刻を取得できる。同じように、パターン3ではtargetからrootへのパスを辿る際に途中でsourceを見つけ、その時点で取得できたデータから最新の座標変換を計算できる時刻を取得できる。

1と4との違いについて説明する。図\ref{fig:same-parent}はパターン4の状況を表し、pがsourceとtarget共通の祖先のフレーム、rがrootフレーム、$t_1$、$t_2$、$t_3$がそれぞれp$\rightarrow$r、s$\rightarrow$p、t$\rightarrow$pの座標変換を取得できる最新の時刻を表す。sourceフレームからtargetフレームへの最新の座標変換を計算できる時刻は$\min(t_2, t_3)$なので$t_1$は無視できる。このため、sourceからrootへ辿る時にアクセスしたフレームを記録しておき、次にtargetからrootへ辿る時に記録を元にしてrootでない共通の祖先を見つけられれば、targetからrootへ辿る処理を省略できる。

\textit{getLatestCommonTime}ではまず、sourceフレームからルートフレームへのエッジをたどり、各エッジで最新の座標変換を提供できる時刻の最小値を取る(\ref{code:getLatest-s-r}〜\ref{code:getLatest-s-r2}行目)。これにより、sourceフレームからルートフレームへの最新の座標変換を取得できる時刻を取得できる。sourceフレームからルートフレームへのエッジをたどるときに各エッジの情報をlct\_cache変数に記録しておく(\ref{code:getLatest-add-history}行目)。\ref{code:getLatest-t-is-p-of-s}行目は上述のパターン2の、targetがsourceの祖先である場合の処理を示している。

続いて、targetフレームからルートフレームへのエッジをたどり、同じようにしてtargetフレームからルートフレームへの最新の座標変換を取得できる時刻を取得する(\ref{code:getLatest-t-r}〜\ref{code:getLatest-t-r2}行目)。\ref{code:getLatest-s-and-t-have-parent}行目はパターン4の、sourceとtargetがroot以外にも同じ祖先を持つ場合の処理を示し、targetフレームからルートフレームへ辿る処理を省略している。\ref{code:getLatest-s-is-p-of-t}行目は上述のパターン3の、targetがsourceの祖先である場合の処理を示している。

最後に、sourceフレームからルートフレームへの最新の座標変換を取得できる時刻、targetフレームからルートフレームへの最新の座標変換を取得できる時刻から、sourceフレームからtargetフレームへの最新の座標変換を取得できる時刻を取得する(\ref{code:getLatest-min} 〜\ref{code:getLatest-min2}行目)。上述のパターン4に対応するために、lct\_cache変数内の記録から最新の座標変換を取得できる時刻を取得している。

sourceフレームからtargetフレームへの最新の座標変換を計算できる時刻を取得できたため、処理は\textit{lookupTransform}に戻り、取得できた時刻はtime変数に代入される。

\textit{lookupTransform}においても、\textit{getLatestCommonTime}と同じように各パターンへの最適化ができる。上述のパターン2と3ではrootまで辿ること無しにその時点で取得できた座標変換情報を利用できる。パターン4においては最適化は行わず、パターン1と同じようにsourceフレームからルートフレームへの座標変換と、ルートフレームからtargetフレームへの座標変換を掛け合わせる。

まず、sourceフレームからルートフレームへのエッジをたどり、sourceフレームからルートフレームへの座標変換を計算する(\ref{code:lookupTrans-s-r}〜\ref{code:lookupTrans-s-r2}行目)。\ref{code:lookupTrans-s-p-of-t}行目は上述のパターン2の、targetがsourceの祖先である場合の処理を示している。

続いて、targetフレームからルートフレームへのエッジをたどり、targetフレームからルートフレームへの座標変換を計算する(\ref{code:lookupTrans-t-r}〜\ref{code:lookupTrans-t-r2}行目)。\ref{code:lookupTrans-t-p-of-s}行目は上述のパターン3の、sourceがtargetの祖先である場合の処理を示している。

最後に、sourceフレームからルートフレームへの座標変換にルートフレームからtargetフレームへの座標変換を掛け合わせることにより、sourceフレームからtargetフレームへの座標変換を取得できる。ルートフレームからtargetフレームへの座標変換は、targetフレームからルートフレームへの座標変換の逆変換を取ることにより取得できる。

\textit{lookupTransform}で最新の座標変換情報を得る場合には内部で\textit{getLatestCommonTime}を呼び出しているため、同じフレームに2度アクセスすることになる。

\begin{algorithm}
\caption{lookupTransform} \label{algo:lookupTransform}
\begin{algorithmic}[1]
	\Function {lookupTransform}{target, source, time} \Comment{フレームsourceからフレームtargetへの時刻timeでの座標変換を計算する}
	\If{time == 0} \Comment{time=0を指定すると、最新の座標変換を計算できる時刻を取得する}
	\State{time = getLatestCommonTime(target, source)} \label{code:getLatest}
	\EndIf
	\State source\_trans = $I$ \Comment{$I$は座標変換の単位元}
	\State frame = source
	\State top\_parent = frame
	\While{frame $\neq$ root} \Comment{sourceからrootまで辿る} \label{code:lookupTrans-s-r}
	\State (trans, parent) = frame.getTransAndParent(time)
	\If{frame == target} \Comment{targetがsourceの祖先の場合} \label{code:lookupTrans-t-p-of-s}
	\State \Return source\_trans
	\EndIf 
	\State source\_trans *= trans \Comment{座標変換の掛け合わせ}
	\State top\_parent = frame
	\State frame = parent
	\EndWhile \label{code:lookupTrans-s-r2}
	\State frame = target
	\State target\_trans = $I$
	\While{frame $\neq$ top\_parent} \Comment{targetからrootまで辿る} \label{code:lookupTrans-t-r}
	\State (tarns, parent) = frame.getTransAndParent(time)
	\If{frame == source} \Comment{sourceがtargetの祖先の場合}  \label{code:lookupTrans-s-p-of-t}
	\State \Return (target\_trans)$^{-1}$
	\EndIf 
	\State target\_trans *= trans
	\State frame = parent
	\EndWhile \label{code:lookupTrans-t-r2}
	\State \Return source\_trans * (target\_trans)$^{-1}$ \Comment{sourceからrootへの変換 * rootからsourceへの変換}
	\EndFunction
\end{algorithmic}
\end{algorithm}


\begin{algorithm}
\caption{getLatestCommonTime}
\begin{algorithmic}[1]
	\Function{getLatestCommonTime}{target, source} \Comment{フレームsourceからフレームtargetへの最新の座標変換を計算できる時刻を取得する}
	\State frame = source
	\State common\_time = TIME\_MAX
	\State lct\_cache = [ ] \Comment{sourceからrootへ辿るときの履歴}
	\While{frame $\neq$ root} \Comment{sourceからrootまで辿る} \label{code:getLatest-s-r}
	\State (time, parent) = frame.getLatestTimeAndParent()
	\State common\_time = min(time, common\_time)
	\State lct\_cache.push\_back((time, parent)) \label{code:getLatest-add-history}
	\State frame = parent
	\If{ frame == target } \Comment{ targetがsourceの祖先の時 } \label{code:getLatest-t-is-p-of-s}
	\State \Return common\_time
	\EndIf
	\EndWhile \label{code:getLatest-s-r2}
	\State frame = target
	\State common\_time = TIME\_MAX
	\While{true} \Comment{targetからrootまで辿る} \label{code:getLatest-t-r}
	\State (time, parent) = frame.getLatestTimeAndParent()
	\State common\_time = min(time, common\_time)
	\If{parent \textbf{in}  lct\_cache} \Comment{ sourceとtargetが同じ祖先を持つとき } \label{code:getLatest-s-and-t-have-parent}
	\State common\_parent = parent
	\State \textbf{break}
	\EndIf
	\State frame = parent
	\If{ frame == source } \Comment{ sourceがtargetの祖先の時 } \label{code:getLatest-s-is-p-of-t}
	\State \Return common\_time
	\EndIf
	\EndWhile \label{code:getLatest-t-r2}
	\For{(time, parent) \textbf{in} lct\_cache} \Comment{lookup tree cacheを元に、最新の時刻を取得する} \label{code:getLatest-min}
	\State common\_time = min(common\_time, time)
	\If{parent == common\_parent}
	\State \textbf{break}
	\EndIf
	\EndFor \label{code:getLatest-min2}
	\State \Return common\_time
	\EndFunction
\end{algorithmic}
\end{algorithm}


\section{setTransform}
二つのフレーム間の座標変換情報を更新するには\textit{setTransform}を使う。図\ref{fig:sample-tree}におけるフレームcからフレームbのように直接の親子関係になっているフレーム間の座標変換情報を更新でき、フレームcからフレームaのように直接の親子関係になっていないフレーム間の座標変換情報は更新できない。フレームcからフレームaへの座標変換を更新するにはフレームcからフレームbへの座標変換、及びフレームbからフレームaへの座標変換を更新すればよい。

このインターフェイスを呼び出すことにより新しい座標変換情報がタイムラインに追加される。

\textit{setTransform}の擬似コードをアルゴリズム\ref{algo:setTransform}に示す。

% 図3.1においてc->aと貼り直すとbのdeleteが発生するが、ここではread/writeのみ扱うのでここでは記述しない。

\begin{algorithm}
\caption{setTransform}\label{algo:setTransform}
\begin{algorithmic}[1]
	\Procedure{setTransform}{transform} \Comment{座標変換transformを登録}
	\State frame = getFrame(transform.child\_frame\_id)
	\State frame.insertData(transform)
	\EndProcedure
\end{algorithmic}
\end{algorithm}

\section{問題点}
\subsection*{問題1: ジャイアント・ロック}
上述のように、TFライブラリの木構造で主に使われるインターフェイスは主に\textit{lookupTransform}と\textit{setTransform}である。これらは複数のスレッドからアクセスされるので並行性制御を行う必要があるが、TFライブラリではmutexオブジェクトを用いて木構造全体を保護している。このため、一つのスレッドが木構造にアクセスしている際は他のスレッドは木構造にアクセスできないアルゴリズムとなっている。複数スレッドが木構造の別々の部分にアクセスするケース、及び複数スレッドが木構造の同じ部分のアクセスしているが全て読み込みアクセスのケースなど、排他制御が必要ではないケースにおいてもアクセスが逐次化されおり、マルチコアが常識となっている現代ではスループットやレイテンシに問題が生じる可能性がある。

\begin{figure}[h] 
\centering
\includegraphics[width=10cm]{snake-sim}	
\caption{シミュレータ上で動作するヘビ型ロボット}
\label{fig:snake-sim}
\end{figure}

\begin{figure}[h] 
\centering
\includegraphics[width=4cm]{snake}	
\caption{ヘビ型ロボットにおける木構造}
\label{fig:snake}
\end{figure}


例えば、ヘビ型ロボットの各関節をTFライブラリで管理する場合について考える。ヘビ型ロボットとはは生物のヘビを模倣したロボットであり、ヘビのように関節を動かすことにより狭い場所の点検や災害現場での被災者探索に用いられている。図\ref{fig:snake-sim}はシミュレータ上で操作する関節数が30のヘビ型ロボットを示す~\cite{snake-sim}。
%TODO ヘビ型ロボットの掲載許可をいただく
% 地面に接触するリンクの位置情報を用いたヘビ型ロボットの自己位置推定
% ヘビ型ロボットのオドメトリと遠隔操作支援

関節の数が1000あり、各関節をフレームとしてTFに登録する場合には図\ref{fig:snake}のように比較的巨大な木構造になる。§\ref{section:tf-interface}にて述べたように、TFの木構造へのアクセスパターンは主に木構造の一部のフレーム間の座標変換情報のみ更新するスレッドと、木構想の一部のフレーム間の座標変換情報のみ取得するスレッドがそれぞれ複数ある状況なので、それぞれのスレッドが木構造の別の部分にアクセスするにもかかわらず、一つのスレッドが木構造にアクセスするたびに木構造全体がロックされてしまいパフォーマンスに問題が生じる可能性がある。

\subsection*{問題2a: データの鮮度}
\label{section:prob-2a}
木構造が図\ref{fig:sample-tree}、タイムラインが図\ref{fig:general-timeline}の状況において、\textit{lookupTransform}を用いてフレームcからフレームdへの最新の座標変換を計算する時には時刻Aの時点での各フレーム間の座標変換データを用いる。この時、b$\rightarrow$aにおいては最新のデータを用いるが、c$\rightarrow$bにおいては最新のデータと一つ前のデータから線形補間されるデータを用いている。d$\rightarrow$aにおいては最新のデータ$\theta$ではなく一つ前のデータ$\alpha$とそのもう一つ前のデータ$\beta$から線形補間されるデータを用いている。このように、\textit{lookupTransform}は二つのフレーム間の座標変換の計算において、フレーム間のエッジの全てにおいて座標変換データを提供できる時刻についての座標変換を計算するという仕様のため、b$\rightarrow$aのように座標変換情報の登録が遅れるとそれに足を引っ張られてしまい、最新の座標変換データが使われなくなるという問題がある。同時刻のデータを元に座標変換を計算するためデータの同期性はあるが、最新の座標変換データを使わないためデータの鮮度は失われる。これにより、ロボットの制御や自己位置推定に問題が生じる可能性がある。現在、TFライブラリには最新の座標変換データをもとにフレーム間の座標変換計算をするインターフェイスは無い。
また、次に説明するようにこの仕様によって座標系間の位置関係があまり変わらない場合についても頻繁にデータを登録する必要があり、無駄な処理が発生する。

\begin{figure}[h] 
\centering
\includegraphics[width=10cm]{need-many-update}
\caption{不必要な更新が必要な例}
\label{fig:need-many-update}
\end{figure}

図\ref{fig:need-many-update}において、b$\rightarrow$aはあまり座標系間の位置関係が変わらないために座標変換はあまり更新されないが、d$\rightarrow$aの座標変換は頻繁に更新されるケースについて考える。座標変換の計算においてb$\rightarrow$aとd$\rightarrow$aのデータを用いる場合、既存のTFでは「ある時刻の座標変換データが保存されているか線形補間で取得できる時に、その時刻の座標変換データを提供できると見做す」という仕様により、b$\rightarrow$aの更新が遅いためにd$\rightarrow$aでは過去の鮮度の低い$\beta$と$\theta$から座標変換の計算をしなくてはならない。これを避けるため、既存のTFではあまり座標系間の位置関係が変わらないb$\rightarrow$aにおいても、一定周期で同じ座標変換情報を登録する必要がある。このように、既存のTFでは座標変換情報が変わらないにもかかわらず一定周期で同じ座標変換情報を登録する必要があり、余計な負荷がかかっている。

\subsection*{問題2b: データの一貫性}
問題2aの解決策として、最新の座標変換データをもとにフレーム間の座標変換計算をするインターフェイスを提供するだけでは不十分である。これは、複数の座標変換データを登録している途中に最新の座標変換データをもとにフレーム間の座標変換計算をすると、ユーザーが期待するデータの一貫性がなくなる可能性があるからである。

\chapter{提案手法}
本研究では、データベースの並行性制御法における細粒度ロッキング法及び2PLを適用し、これらの問題を解決する。

\section{細粒度ロックの導入}
\label{section:intro-high-gran-lock}

前述した問題1については、データベースの並行性制御法における細粒度ロッキング法を適用して解決する。

図\ref{fig:giant-lock}の木構造においてスレッド1が\textit{lookupTransform}を用いてフレームcからフレームaへの座標変換の計算、スレッド2が\textit{setTransform}を用いてフレームdからフレームaへの座標変換を更新する場合について考える。ここで、スレッド1はフレームcとフレームbのデータの読み込み、スレッド2はフレームdのデータの書き込みを行うため、スレッド$i$のデータ$x$に対する読み込み操作を$r_i(x)$、スレッド$i$のデータ$x$に対する書き込み操作を$w_i(x)$と表記すると、スレッド1の操作は$r_1(c)r_1(b)$、スレッド2の操作は$w_2(d)$と表記できる。


\begin{figure}[h] 
\centering
\includegraphics[width=4cm]{gaint-lock}	
\caption{Giant lock}
\label{fig:giant-lock}
\end{figure}


TFライブラリでは木構造へのアクセスをする際、木構造全体をジャイアント・ロックする。これにより、図\ref{fig:giant-lock}の点線枠部分が保護される。図\ref{fig:g-lock-time}はスレッド1の処理中にスレッド2の処理が開始した時のスケジュールを図示している。セルが実行中の処理を表し、セルの端のGlockとGunlockは木構造へのジャイアントロック、アンロックを表す。スレッド2の処理が開始した時、スレッド1が木構造をジャイアントロックしているため、スレッド2はスレッド1の処理が完了し木構造のロックが外されるまで待機する必要がある。スレッド1がアクセスするデータとスレッド2がアクセスするデータは異なるため、より細かくロックする範囲を指定できる方法があればスレッド2がスレッド1の処理の完了を待つ必要がなくなる。


\begin{figure}[h] 
\centering
\includegraphics[width=7cm]{g-lock-time.png}	
\caption{ジャイアントロックにおけるスケジュール}
\label{fig:g-lock-time}
\end{figure}

そこで、本研究ではデータベースの並行性制御法における細粒度ロッキング法を適用する。
細粒度ロッキング法ではアクセスするデータにのみロックを確保し、さらにロックの種類を読み込みロックと書き込みロックに分ける。
複数のスレッドが同じデータにアクセスする際に発生するデータ競合を避けるために、排他制御では一つのスレッドからのみデータにアクセスできるようにするため、ロックを確保する。しかしながら、複数のスレッドが同じデータにアクセスする際、データの読み込みのみ行うのであればデータ競合は発生しない。そこで、データの読み込みのみを行う時には読み込みロック、データの書き込みを行う時には書き込みロックを使い、次のルールを設ける。

\begin{itemize}
 \item 読み込みを行う前に読み込みロック、書き込みを行う前に書き込みロックを確保する必要がある
 \item ロックされていないデータには読み込みロック、及び書き込みロックを確保できる
 \item すでに読み込みロックされたデータにも他のスレッドが読み込みロックを確保することができる
 \item すでに読み込みロックされたデータには他のスレッドが書き込みロックを確保することはできない
 \item すでに書き込みロックされたデータには他のスレッドは読み込みロックも書き込みロックも確保することはできない
\end{itemize}


\begin{table}[h!]
\centering
\begin{tabular}{ | m{1cm} | m{1cm} | m{1cm} | } 
  \hline
  & $rl_i(x)$ & $wl_i(x)$ \\ 
  \hline
  $rl_j(x)$ & $\circ$ & $\times$ \\ 
  \hline
  $wl_j(x)$ &  $\times$ & $\times$ \\ 
  \hline
\end{tabular}	
\caption{各ロックの互換性}
\label{table:lock-table}
\end{table}

このルールを表\ref{table:lock-table}にまとめる。表中ではスレッド$i$のデータ$x$に対する読み込みロック操作を$rl_i(x)$、スレッド$i$のデータ$x$に対する書き込みロック操作を$wl_i(x)$と表記する。表は一行目がスレッド$i$によってロックが確保された状態を表し、その状態に読み込みロック、または書き込みロックをスレッド$j(i \neq j)$が確保できるかどうかを2、3行目で表している。$\circ$はすでにロックが確保されていても別のスレッドがロックを確保できることを表し、$\times$はそうでないことを表す。例えば、2行2列目はすでに$rl_i(x)$が確保されていても$rl_j(x)$が確保できることを表し、2行3列目はすでに$wl_i(x)$が確保されていると$rl_j(x)$は確保できないことを表す。

\begin{figure}[h] 
\centering
\includegraphics[width=7cm]{high-gran-lock}
\caption{細粒度ロッキング}
\label{fig:high-gran-lock}
\end{figure}


\begin{figure}[h] 
\centering
\includegraphics[width=7cm]{high-gran-time}
\caption{細粒度ロックにおけるスケジュール}
\label{fig:high-gran-time}
\end{figure}



細粒度ロッキングを用いた場合のスレッド1、スレッド2の保護範囲は図\ref{fig:high-gran-lock}、スケジュールは図\ref{fig:high-gran-time}で表せる。図\ref{fig:high-gran-time}ではスレッド$i$がデータ$x$を読み込みアンロック、書き込みアンロックする時にはそれぞれ$ru_i(x), wu_i(x)$と表記される。

スレッド1の実行中にスレッド2の処理が開始しても、図\ref{fig:high-gran-lock}で表されるようにスレッド1とスレッド2でアクセスするデータは異なるため、スレッド2はスレッド1の処理完了を待機する必要がなくなる。このスケジュールは各操作を時系列順に表記することにより

\begin{equation}
rl_1(c)r_1(c)wl_2(d)w_2(d)ru_1(c)rl_1(b)r_1(b)wu_2(d)ru_1(b)
\end{equation}
と書ける。


\begin{figure}[h] 
\centering
\includegraphics[width=5cm]{two-read-lock}
\caption{二つの読み込みロック}
\label{fig:two-read-lock}
\end{figure}


\begin{figure}[h] 
\centering
\includegraphics[width=7cm]{two-read-lock-time}
\caption{二つの読み込みロックにおけるスケジュール}
\label{fig:two-read-lock-time}
\end{figure}

スレッド1の処理の途中に、\textit{lookupTransform}を用いてフレームdのデータを読み込むスレッド3が開始するケースについて考える。細粒度ロッキングを用いた場合のスレッド1とスレッド3の保護範囲は図\ref{fig:two-read-lock}で、スケジュールは図\ref{fig:two-read-lock-time}で表記される。スレッド1にてデータbに対して読み込みロックを確保するときすでにスレッド3がbを読み込みロックしているが、表\ref{table:lock-table}が表すようにすでに読み込みロックが確保されていても他のスレッドが読み込みロックをかけることができる。このスケジュールは
\begin{equation}
	rl_1(c)r_1(c)rl_3(b)r_3(b)ru_1(c)rl_1(b)r_1(b)ru_3(b)ru_1(b)
\end{equation}
と書ける。


このように、細粒度ロッキング法ではデータごとにロックをし、さらに読み込みロック・書き込みロックと区別をつけることにより並行性を上げることができる。

細粒度ロックを実装した\textit{lookupTransform}、\textit{getLatestCommonTime}、\textit{setTransform}の擬似コードをそれぞれアルゴリズム\ref{algo:lookupTransform2}、アルゴリズム\ref{algo:getLatestCommonTime2}、アルゴリズム\ref{algo:setTransform2}にて示す。アルゴリズム\ref{algo:lookupTransform2}にて\ref{code:lookup-fine-rlock}〜\ref{code:lookup-fine-runlock}行目にて示されるように、木構造全体ではなくframe.rLock()で一つのフレームの読み込みロックのみ確保することにより、細粒度ロックを実装している。アルゴリズム\ref{algo:setTransform2}も同様に、一つのフレームにのみframe.wLock()により書き込みロックを確保することにより細粒度ロックを実装している。

\begin{algorithm}
  \caption{細粒度ロックを実装したlookupTransform}\label{algo:lookupTransform2}
\begin{algorithmic}[1]
	\Function {lookupTransform}{target, source, time} 
	\If{time == 0}
	\State{time = getLatestCommonTime(target, source)}
	\EndIf
	\State source\_trans = $I$
	\State frame = source
	\State top\_parent = frame
	\While{frame $\neq$ root}
	\State frame.rLock() \label{code:lookup-fine-rlock} \Comment{読み込みロックを確保}
	\State (trans, parent) = frame.getTransAndParent(time) 
	\State frame.rUnlock() \label{code:lookup-fine-runlock} \Comment{読み込みロックを解放}
	\If{frame == target} 
	\State \Return source\_trans
	\EndIf 
	\State source\_trans *= trans
	\State top\_parent = frame
	\State frame = parent
	\EndWhile
	\State frame = target
	\State target\_trans = $I$
	\While{frame $\neq$ top\_parent}
	\State frame.rLock()
	\State (tarns, parent) = frame.getTransAndParent(time)
	\State frame.rUnlock()
	\If{frame == source}
	\State \Return (target\_trans)$^{-1}$
	\EndIf 
	\State target\_trans *= trans
	\State frame = parent
	\EndWhile
	\State \Return source\_trans * (target\_trans)$^{-1}$
	\EndFunction
\end{algorithmic}
\end{algorithm}

\begin{algorithm}
\caption{細粒度ロックを実装したgetLatestCommonTime} \label{algo:getLatestCommonTime2}
\begin{algorithmic}[1]
	\Function{getLatestCommonTime}{target, source} 
	\State frame = source
	\State common\_time = TIME\_MAX
	\State lct\_cache = [ ] 
	\While{frame $\neq$ root}
	\State frame.rLock()
	\State (time, parent) = frame.getLatestTimeAndParent()
	\State frame.rUnLock()
	\State common\_time = min(time, common\_time)
	\State lct\_cache.push\_back((time, parent))
	\State frame = parent
	\If{ frame == target } 
	\State \Return common\_time
	\EndIf
	\EndWhile
	\State frame = target
	\State common\_time = TIME\_MAX
	\While{true}
	\State frame.rLock()
	\State (time, parent) = frame.getLatestTimeAndParent()
	\State frame.rUnLock()
	\State common\_time = min(time, common\_time)
	\If{parent \textbf{in}  lct\_cache}
	\State common\_parent = parent
	\State \textbf{break}
	\EndIf
	\State frame = parent
	\If{ frame == source } 
	\State \Return common\_time
	\EndIf
	\EndWhile
	\For{(time, parent) \textbf{in} lct\_cache}
	\State common\_time = min(common\_time, time)
	\If{parent == common\_parent}
	\State \textbf{break}
	\EndIf
	\EndFor
	\State \Return common\_time
	\EndFunction
\end{algorithmic}
\end{algorithm}

\begin{algorithm}
\caption{細粒度ロックを実装したsetTransform}\label{algo:setTransform2}
\begin{algorithmic}[1]
	\Procedure{setTransform}{transform}
	\State frame = getFrame(transform.child\_frame\_id)
	\State frame.wLock() \Comment{書き込みロックを確保}
	\State frame.insertData(transform)
	\State frame.wUnlock() \Comment{書き込みロックを解放}
	\EndProcedure
\end{algorithmic}
\end{algorithm}



\section{2PLの導入}
\label{section:intro-2pl}

前述した問題2a、2bについては、複数の座標変換のデータをatomicに取得するインターフェース(\textit{lookupLatestTransformXact})、及び複数の座標変換の最新のデータをatomicに更新するインターフェース(\textit{setTransformsXact})を提供して解決する。

\subsection{データの鮮度の確保}
まず、二つのフレーム間の座標変換に線形補間を行わずにフレーム間のエッジの最新の座標変換データを使うインターフェイスとして\textit{lookupLatestTransform}を導入する。これは木構造が図\ref{fig:sample-tree}、タイムラインが図\ref{fig:general-timeline}における状況でフレームcからフレームdへの座標変換は次のように計算される。

\begin{enumerate}
	\item フレームcから木構造のルートフレームへエッジをたどりながら、各フレーム間の最新の座標変換を掛け合わせてフレームから木構造のルートフレームへの座標変換を計算する。ここでは、フレームcから木構造のルートフレームへのエッジはc$\rightarrow$b、b$\rightarrow$aとなり、それぞれの座標変換は最新のものを使う。
	\item 同じように、フレームdから木構造のルートフレームへエッジをたどりながら、各フレーム間の最新の座標変換を掛け合わせてフレームから木構造のルートフレームへの座標変換を計算する。
	\item フレームcから木構造のルートフレームへの座標変換と、木構造のルートフレームからフレームdへの座標変換を掛け合わせる。木構造のルートフレームからフレームdへの座標変換はフレームdから木構造のルートフレームへの座標変換の逆変換から得られる。
\end{enumerate}

\begin{figure}[h] 
\centering
\includegraphics[width=12cm]{lookupLatestTransform}
\caption{lookupLatestTransformで取得するデータ}
\label{fig:lookupLatestTransform}
\end{figure}

\textit{lookupLatestTransform}にて取得する座標変換データは、図\ref{fig:lookupLatestTransform}のように図示できる。

\textit{lookupLatestTransform}を用いることにより、§\ref{section:prob-2a}で説明した余計な負荷がかかる問題も解決できる。これは、\textit{lookupLatestTransform}は\textit{lookupTransform}とは異なり、最新の座標変換のみを見るため必要な時にのみ座標変換の更新をすればよく、§\ref{section:prob-2a}で説明した余計な負荷がかかることはなくなるからである。

\textit{lookupLatestTransform}の疑似コードをアルゴリズム\ref{algo:lookupLatestTransform}にて示す。

\begin{algorithm}
  \caption{lookupLatestTransform}\label{algo:lookupLatestTransform}
\begin{algorithmic}[1]
	\Function {lookupLatestTransform}{target, source}
	\State source\_trans = $I$
	\State frame = source
	\State top\_parent = frame
	\While{frame $\neq$ root}
	\State frame.rLock()
	\State (trans, parent) = frame.getLatestTransAndParent() 
	\State frame.rUnlock()
	\If{frame == target}
	\State \Return source\_trans
	\EndIf
	\State source\_trans *= trans
	\State top\_parent = frame
	\State frame = parent
	\EndWhile
	\State frame = target
	\State target\_trans = $I$
	\While{frame $\neq$ top\_parent}
	\State frame.rLock()
	\State (tarns, parent) = frame.getTransAndParent(time)
	\State frame.rUnlock()
	\If{frame == source}
	\State \Return (target\_trans)$^{-1}$
	\EndIf
	\State target\_trans *= trans
	\State frame = parent
	\EndWhile
	\State \Return source\_trans * (target\_trans)$^{-1}$
	\EndFunction
\end{algorithmic}
\end{algorithm}


% でも、TFは同期を取ったほうがいいとされるからああいう設計にしたんじゃ?最新のデータだけ保存するようにしてればよかったじゃん
%TODO ここ突っ込まれたら痛いなああああ、ここの回答は「この研究を通して既存のTFもそうなってくれると嬉しいですね」程度のコメントで

\subsection{データの一貫性の確保}
\textit{lookupLatestTransform}を新たに提供することにより、暗黙的な線形補間をさけ、最新の座標変換データをもとにしたフレーム間の座標変換が計算できる。しかし、これには次のようなケースでは問題となる。

\begin{figure}[h] 
\centering
\includegraphics[width=12cm]{coming-same-time}
\caption{同時に座標変換が登録されるケース}
\label{fig:coming-same-time}
\end{figure}


図\ref{fig:coming-same-time}は木構造が図\ref{fig:sample-tree}の時にフレームbからフレームaへの座標変換と、フレームdからフレームaへの座標変換が同時刻に登録されるケースでのタイムラインを表す。
b$\rightarrow$aとa$\rightarrow$cのデータを用いてフレームbからフレームcへの座標変換を計算する際、ユーザーはb$\rightarrow$aとd$\rightarrow$aのデータについては同時刻のものを使うことを期待する。しかしながら、\textit{lookupLatestTransform}を使うとユーザーの期待に反して図\ref{fig:coming-same-time}のように$\theta$がまだ登録されていない中間状態のタイムラインを観測し、$\alpha$と$\beta$を元に座標変換してしまうことがある。これは、複数の座標変換の登録において\textit{setTransform}を複数呼び出す際、 全ての座標変換が登録できていない状態で\textit{lookupLatestTransform}が木構造にアクセスできることに起因する。従来の\textit{lookupTransform}ではフレーム間のエッジの全てにおいて座標変換データを提供できる時刻についての座標変換を計算するという仕様のため、このような問題は発生しなかった。

% ただの細粒度ロックだとSerializabilityがなくなる、けどgiant lockだとパフォーマンスがまずいという話を説明する必要はある?

そこで、複数の座標変換を2PLによってatomicに木構造に登録する\textit{setTransformsXact}を提供し、また
\textit{lookupLatestTransform}を2PLを使うように変更した\textit{lookupLatestTransformXact}を提供する。
2PL~\cite{2PL}とは、複数のデータに対するロック・アンロックを二つのフェーズに分けることによって並行処理の結果が直列処理と同じ結果になることを保証する、データベースにおける並行性制御法である。このような性質は、並行性制御法においてはSerializability~\cite{2PL}と呼ばれる。

2PLによって並行性制御をしたときの\textit{setTransformsXact}と\textit{lookupLatestTransformXact}の動作について説明する。スレッド1が\textit{setTransformsXact}を用いてb$\rightarrow$a、d$\rightarrow$aの情報を更新し、スレッド2が\textit{lookupLatestTransformXact}を用いてb$\rightarrow$a、d$\rightarrow$aの情報を元にフレームbからフレームdへの座標変換を計算し、スレッド1の処理中にスレッド2の処理が開始するケースについて考える。それぞれのスレッドの処理は$w_1(b)w_1(d)$、$r_2(b)r_2(d)$と表現できる。

\begin{figure}[h] 
\centering
\includegraphics[width=12cm]{setTransforms1}
\caption{setTransformsXactとlookupLatestTransformXact 1}
\label{fig:setTransforms1}
\end{figure}

図\ref{fig:setTransforms1}はスレッド1で$w_1(b)$をする前に$wl_1(b)$をした時の様子を表している。この状態でスレッド2の処理が始まると、$r_2(b)$をするために$rl_2(b)$を確保する必要があるが、まだ$wl_1(b)$がかけられているためにロックが外されるまで待つ必要がある。

\begin{figure}[h] 
\centering
\includegraphics[width=12cm]{setTransforms2}
\caption{setTransformsXactとlookupLatestTransformXact 2}
\label{fig:setTransforms2}
\end{figure}

図\ref{fig:setTransforms2}はスレッド1で$w_1(b)$が完了してデータ$\alpha$が登録され、$w_1(d)$をする前に$wl_1(d)$をした時の様子を表している。2PLでは複数のロックを確保し、必ず全てのロックを取り終えてからアンロックをしていく。このため、bへのロックはまだ解放されておらずスレッド2は待機する必要がある。


\begin{figure}[h] 
\centering
\includegraphics[width=12cm]{setTransforms3}
\caption{setTransformsXactとlookupLatestTransformXact 3}
\label{fig:setTransforms3}
\end{figure}


図\ref{fig:setTransforms3}はスレッド1の処理が完了してデータ$\theta$が登録され、スレッド1によるフレームb,dへのロックが開放され、スレッド2が$r_2(b)$と$r_2(d)$をするために$rl_2(b)$と$rl_2(d)$のロックを確保した時の様子である。2PLによりスレッド1の処理が完了してからスレッド2は木構造へアクセスできるため、スレッド2が中間の状態を観測することはなくなる。

この一連のスケジュールは
\begin{equation}
	wl_1(b)w_1(b)wl_1(d)w_1(d)wu_1(b)wu_1(d)rl_2(b)r_2(b)rl_2(d)r_2(d)ru_2(b)ru_2(d)
\end{equation}
と表記できる。

\subsection{Deadlockの回避}
さて、2PLの導入によって複数のデータに対する読み込み・書き込みがatomicに行えるようになったが、複数のデータに対してロックを取ることによりdeadlockの可能性が生じる。


\begin{figure}[h] 
\centering
\includegraphics[width=7cm]{deadlock}
\caption{deadlock}
\label{fig:deadlock}
\end{figure}

図\ref{fig:deadlock}のような木構造において、スレッド1がフレームcからフレームdへの座標変換を\textit{lookupLatestTransformXact}を用いて計算し、スレッド2が\textit{setTransformsXact}を用いてd$\rightarrow$a及びa$\rightarrow$eの座標変換を更新する場合について考える。それぞれのスレッドの操作は$r_1(c)r_1(b)r_1(a)r_1(d)$、$w_2(d)w_2(a)$と表記できる。

図\ref{fig:deadlock}はスレッド1がa, b, cの読み込みロック、スレッド2がdの書き込みロックを確保した状態を表している。ここで、スレッド1は次にdの読み込みロックを確保したいがすでにスレッド2がdを書き込みロックしているためロックの解放を待機する必要がある。スレッド2は次にaの書き込みロックを確保したいがすでにスレッド1がaを読み込みロックしているために待機する必要がある。二つのスレッドがお互いのロック解放を待ち続けるため、deadlockとなる。これはスレッド2が木構造のルートフレームへ登る方向にロックをかけているのに対し、スレッド1は逆に一時的に木構造を下る方向にロックをかけていることに起因する。

% 要素のreorderによって防げるけどちょいとむずそう

そこで、我々はdeadlockを防ぐ方法としてNoWait~\cite{nowait}を採用した。これは、書き込みロックを2つ以上かけようとしたときにすでにデータがロックされていたら、保持しているロックを全て解放し最初から処理をやり直す手法である。これにより、書き込みロックを2つ以上しているスレッドがロックの解放を待機することがなくなり、deadlockは発生しない。また、我々の手法ではcontention regulationとして保持しているロックを全て解放して1ミリ秒経過してから最初から処理をやり直す。

\begin{figure}[h] 
\centering
\includegraphics[width=10cm]{dirty-read}
\caption{Dirty Read}
\label{fig:dirty-read}
\end{figure}


NoWaitによって処理のやり直しが発生するため、\textit{setTransformsXact}では書き込みを行うタイミングに注意する必要がある。T1が\textit{setTransformsXact}を用いて$w_1(a)w_1(b)$、T2が\textit{lookupLatestTransformXact}を用いて$r_2(b)r_2(a)$を実行する時、図\ref{fig:dirty-read}のようなスケジュールになったケースについて考える。T1が$a$の書き込みを終えた後に$b$への書き込みロックを取ろうとするが、すでにT2によって$b$への読み込みロックは取られているので、NoWaitによって$a$へのロックを外してから処理をやり直す。図中の$\times$印は全てのロックを外し、処理を最初からやり直す事を表している。

T1による処理のやり直しの前にT2が$a$を読み込んでしまうと、T2は$a$についてはT1による更新後のデータ、$b$についてはT1による更新前のデータを読んでしまい、並行処理の結果が直列処理と同じ結果にならなくなってしまう。この問題は、トランザクション理論においてはDirty readと呼ばれる。Dirty readを避けるため、我々の手法では全ての書き込みロックが確保できてから座標変換の書き込みを行うようにした。これにより、書き込みが一部行われた状態を読み込み専用スレッドが観測することはなくなる。

\textit{lookupLatestTransformXact}、\textit{setTransformsXact}の擬似コードをそれぞれアルゴリズム\ref{algo:lookupLatestTransformXact}、\ref{algo:setTransformsXact}に示す。\textit{lookupLatestTransform}とは異なり、\textit{lookupLatestTransformXact}では読み込みロックを取った後は\ref{code:lookupXact-add}行目のように読み込みロックのリストに追加し、\ref{code:lookupXact-unlockall}行目のように座標変換を取得した後にリスト内の全てのフレームのロックを解放することにより、2PLを実装している。\textit{setTransformsXact}も同じように書き込みロックが取れたら書き込みロックのリストに追加し、座標変換の登録ができてからリスト内の全てのフレームのロックを解放する。\ref{code:setXact-trylock}行目のframe.tryWLock()はもし他のスレッドが読み込みロックも書き込みロックも確保していない場合には書き込みロックをし、trueを返す。もし他のスレッドがロックを確保していた場合にはfalseを返し、書き込みロックの確保失敗を表す。書き込みロックの確保が失敗失敗したらリスト内の全てのフレームのロックを解放し、1ms待機してから処理をやり直す。

\begin{algorithm}
  \caption{lookupLatestTransformXact}\label{algo:lookupLatestTransformXact}
\begin{algorithmic}[1]
	\Function {lookupLatestTransformXact}{target, source}
	\State rlock\_list = [ ]
	\State source\_trans = $I$
	\State frame = source
	\State top\_parent = frame
	\While{frame $\neq$ root}
   \State frame.rLock()
	\State rlock\_list.push\_back(frame) \Comment{読み込みロックのリストに追加} \label{code:lookupXact-add}
	\State (trans, parent) = frame.getLatestTransAndParent() 
	\If{frame == target}
	\For{f in rlock\_list} \Comment{リスト内の全ての要素のロックを解放}
	\State f.rUnlock()
	\EndFor
	\State \Return source\_trans
	\EndIf
	\State source\_trans *= trans
	\State top\_parent = frame
	\State frame = parent
	\EndWhile
	\State frame = target
	\State target\_trans = $I$
	\While{frame $\neq$ top\_parent}
   \State frame.rLock()	
	\State rlock\_list.push\_back(frame) \Comment{読み込みロックのリストに追加}
	\State (tarns, parent) = frame.getTransAndParent(time)
	\If{frame == source}
	\For{f in rlock\_list} \Comment{リスト内の全ての要素のロックを解放}
	\State f.rUnlock()
	\EndFor
	\State \Return (target\_trans)$^{-1}$
	\EndIf
	\State target\_trans *= trans
	\State frame = parent
	\EndWhile
   \For{f in rlock\_list} \label{code:lookupXact-unlockall}  \Comment{リスト内の全ての要素のロックを解放}
   \State f.rUnlock()
   \EndFor	
	\State \Return source\_trans * (target\_trans)$^{-1}$
	\EndFunction
\end{algorithmic}
\end{algorithm}


\begin{algorithm}
\caption{setTransformsXact}\label{algo:setTransformsXact}
\begin{algorithmic}[1]
	\Procedure{setTransformsXact}{transforms}
	\State wlock\_list = [ ] \label{op1}
	\For{trans in transforms}
	\State frame = getFrame(trans.child\_frame\_id)
	\State lock\_success = frame.tryWLock() \Comment{書き込みロックの確保を試みる} \label{code:setXact-trylock}
	\If{lock\_success}
	\State wlock\_list.push\_back(frame)
	\Else
	\For{f in wlock\_list} \label{code:setXact-unlockall}  \Comment{リスト内の全ての要素のロックを解放}
   \State f.wUnlock()
   \EndFor
	\State sleep 1ms 
	\State \textbf{goto} \ref{op1} \Comment{処理をやり直す}
	\EndIf
	\EndFor
	\For{trans in transforms} \Comment{Dirty readを避けるため、全てのwlockが確保できてから書き込み}
	\State frame = getFrame(trans.child\_frame\_id)
	\State frame.insertData(trans)
	\EndFor
	\For{f in wlock\_list} \Comment{リスト内の全ての要素のロックを解放}
   \State f.wUnlock()
   \EndFor
	\EndProcedure
\end{algorithmic}
\end{algorithm}

\chapter{評価}
TFライブラリに細粒度を実装した\textit{lookupTransform}・\textit{setTransform}、及び最新の複数のデータをatomicに取得・更新できる\textit{lookupLatestTransformXact}・\textit{setTransformsXact}を以下の指標で評価した。

\begin{enumerate}
	\item スループット: 一秒間に何回操作(\textit{lookupTransform}及び\textit{setTransforms})ができたか
	\item レイテンシ: 操作の応答時間
	\item データの鮮度: \textit{lookupTransform}においてアクセスした各座標変換データのタイムスタンプの新しさを表す。アクセスした各座標変換データのタイムスタンプの平均とアクセス時の時刻の差をdelayとし、delayが少ない方がデータの鮮度が高いとみなす。
	\item データの同期性(\textit{lookupLatestTransformXact}のみ): \textit{lookupLatestTransformXact}においてアクセスした各座標変換データのタイムスタンプの同一性を表す。アクセスした各座標変換データのタイムスタンプの標準偏差を計算し、これが小さい方がデータの同期性があるとみなす。
	\item abort率(\textit{setTransformsXact}のみ): \textit{setTransformsXact}において、NoWaitによって操作をやり直した回数の比率。(やり直した回数 / \textit{setTransformsXact}を呼び出した回数) で求める。
\end{enumerate}

以下、既存手法はold、細粒度ロックを実装した\textit{lookupTransform}・\textit{setTransforms}をsnapshot、2PLを用いて複数のデータをatomicに取得・更新できる\textit{lookupLatestTransformXact}・\textit{setTransformsXact}をlatestと呼称する。

\section{実装}
§\ref{section:intro-tf}で述べたように、TFライブラリはROS上で動作する。ROSはUbuntu上で動作し、Ubuntuのバージョンごとに別のディストリビューションが公開されている。これに合わせ、TFライブラリもROSのディストリビューション毎に提供されている。TFライブラリの実装はGithubのリポジトリ~\cite{ros-geometry2}で公開され、ブランチ毎に各ROSのディストリビューション向けの実装がされている。この対応関係を表\ref{table:ubuntu-ros-branch}に示す。

\begin{table}[ht]
\centering
\begin{tabular}[t]{lcc}
\hline
Ubuntuのバージョン & 対応するROSのディストリビューション & 対応するgeometry2のブランチ名\\
\hline
20.04(Focal) & ROS Noetic Ninjemys & noetic-devel \\
18.04(Bionic) & ROS Melodic Morenia & melodic-devel \\
14.04(Trusty) & ROS Indigo Igloo & indigo-devel \\
\hline
\end{tabular}
\caption{Ubuntuのバージョンと対応するブランチ名}
\label{table:ubuntu-ros-branch}
\end{table}%

このリポジトリのデフォルトブランチはmelodic-develであるが、TFの木構造の並行性制御アルゴリズムはどのブランチでも変わらない。また、ROS2向けのTFライブラリの実装は~\cite{ros2-geometry2}で公開されているが、こちらも木構造の並行性制御アルゴリズムはROS向けのものと変わらない。

このため、本研究ではUbuntu18.04を搭載したマシンにROS Melodic Moreniaをインストールし、TFライブラリのGithubのリポジトリ~\cite{ros-geometry2}のmelodic-develの実装を変更して実験を行った。C++言語で実装をし、1236行分の変更を行った。この実装は~\cite{ogiwara-geometry2}で公開されている。
%src +534 -254
%include +397 -51
%total +5073 -327

\section{実験環境}

実験にはIntel(R) Xeon(R) Platinum 8176 CPU @ 2.10GHzを4つ搭載したサーバを利用する。それぞれのコアは32KB private L1dキャッシュ、1024KB private L2キャッシュを持つ。単一プロセッサの28コアは39MB L3キャッシュを共有し、ハイパー・スレッディングを有効化している。トータルキャッシュサイズはおよそ160MBである。メモリはDDR4-2666が48個接続されており、一つあたりのサイズは32GB、全体のサイズは1.5 TBである。全ての実験において,実行時間は 60秒という安定的な結果が得られる時間を選択した。

\section{ワークロード}

% Applicationはどこにかく?
実験のワークロードは、関節数が多いヘビ型ロボットの関節情報をTFに登録することを想定し、図\ref{fig:snake}のようなフレーム間の座標変換情報が一直線に与えられた構造に複数のスレッドからアクセスし計測を行う。TFライブラリへのアクセスパターンとしては、主に\textit{lookupTransform}のみを複数回呼び出すスレッドと\textit{setTransform}のみを複数回呼び出すスレッドに二分される。このため、\textit{lookupTransform}のみを複数回呼び出すスレッド(読み込み専用スレッド)、及び\textit{setTransform}のみを複数回呼び出すスレッド(書き込み専用スレッド)をそれぞれ複数立ち上げ計測を行う。

実験においては以下のパラメータが存在する。
\begin{enumerate}
  \item thread: 合計スレッド数
  \item joint: フレームの数
  \item read\_ratio: 合計スレッド数のうち、読み込み専用スレッドの割合
  \item read\_len: 読み込み専用スレッドにて一回の\textit{lookupTransform}操作で読みこむフレームの数。joint個のフレームのうち一様分布を元にランダム(つまり、Skew=0)にi番目のフレームが選択され、そこからi+read\_len番目のフレームまでの座標変換が計算される。
  \item write\_len: 書き込み専用スレッドにて一回の操作で座標変換情報を更新するフレームの数。joint個のフレームのうち一様分布を元にランダムにi番目のフレームが選択され、そこからi+read\_len番目のフレームまでの座標変換が更新される。\textit{setTransform}では一度に一個のフレームしか更新できないためwrite\_len回\textit{setTransform}を呼び出す操作を一つの操作とする。
  \item frequency: 各スレッドにて操作を呼び出す周期。操作の呼び出しが完了したのち、1 / frequency秒待機してから再び操作を呼びだす。0に設定すると待機なしで操作を呼び出し続ける。各スレッドが一定の周期にて操作を呼び出すというのは、ROSにおいて一般的なワークロードである。
%  \item opposite\_write: 本実験において、読み込み専用スレッドは図\ref{fig:opposite-write}のように下から上に順に読み込みロックをとる。このフラグをtrueにすると、上から下の方向に書き込みロックをとり、読み込みロックとは逆方向にロックをとる。falseにすると下から上の方向に書き込みロックをとる。
\end{enumerate}

% Joint=1'000'000では変化が出なかったためコメントアウト
%\begin{figure}[h] 
%\centering
%\includegraphics[width=15cm]{opposite-write}
%\caption{opposite\_writeフラグ}
%\label{fig:opposite-write}
%\end{figure}


各実験はそれぞれYCSB-A/B/C~\cite{ycsb}ワークロードについて行った。YCSB-A/B/Cはそれぞれ、読み込み操作と書き込み操作の割合が50:50、95:5、100:0のワークロードを指す。ここでは読み込み専用スレッドの数と書き込み専用スレッドの数の比でそれぞれのワークロードを再現するため、read\_ratioをそれぞれ0.5、0.95、1に設定した。
%Skewについて0

特に記載がない場合はjoint=1000000、read\_len=16、write\_len=16、frequency=0で実験が行われている。また、上述のようにSkewは0に設定して実験を行った。
%TODO joint=100, 60Hzでスレッド数変えたテスト

%Skewについてはきく

\section{YCSB-C}

\begin{figure}[h] 
\centering
\includegraphics[width=15cm]{ycsb-c/opposite-read-throughput}
\caption{YCSB-Cにおけるスレッド数と読み込みスループットの関係}
\label{fig:throughput-c}
\end{figure}

\begin{figure}[h] 
\centering
\includegraphics[width=15cm]{ycsb-c/opposite-read-latency}
\caption{YCSB-Cにおけるスレッド数とレイテンシの関係}
\label{fig:latency-c}
\end{figure}

\begin{figure}[h] 
\centering
\includegraphics[width=15cm]{ycsb-c/opposite-read-latency2}
\caption{YCSB-Cにおけるスレッド数とレイテンシの関係 snapshotとlatestのみ}
\label{fig:latency-c2}
\end{figure}

スループットについては図\ref{fig:throughput-c}のように、oldに比べてsnapshotは最大243倍、latestは最大257倍のスループットとなった。oldと比べ、論理コア数である224倍以上の性能差が出たのは、TFの木構造のC++の実装において各フレームを管理する方法を既存手法から変更したからだと考えられる。

各フレームはC++の実装においてTimeCacheクラスで表現され、TFの木構造中のフレーム群はstd::vector<std::shared\_ptr<TimeCache>~>で管理される。std::shared\_ptr型は自身を参照しているスコープの数をカウンタで管理するため、複数スレッドからstd::shared\_ptr型のデータにアクセスする際にこのカウンタへの読み込み・書き込みが複数スレッドから行われる。これによりキャッシュミス率が増加し性能劣化につながる。これを避けるため、提案手法の実装ではフレーム群はstd::vector<TimeCache*>で管理される。このような実装の違いが論理コア数である224倍以上の性能差につながったと考えられる。

レイテンシについては図\ref{fig:latency-c}、図\ref{fig:latency-c2}のように、どの手法においてもレイテンシとスレッド数が線形比例しているが、oldに比べsnapshot、latestは小さいレイテンシとなった。

スループット、レイテンシのどちらにおいても提案手法が既存手法より優れているのは、既存手法ではジャイアントロックにより操作を並行に行えないが、提案手法では細粒度ロックと2PLによって操作を並行に行えるからだと考えられる。また、latestの方がsnapshotより優れた性能をしてしているのは、§\ref{section:intro-high-gran-lock}で説明したようにsnapshotのlookupTransformでは、§\ref{section:lookupTrnasform}で説明したように木構造を2度読み込む必要があるからだと考えられる。

書き込みが発生しないため、YCSB-Cにおける書き込みスループット、書き込みレイテンシ、データの鮮度、データの同期性、abort率については記述しない。

\section{YCSB-A}
\label{section:ycsb-a}

\begin{figure}[h] 
\centering
\includegraphics[width=15cm]{ycsb-a/opposite-throughput}
\caption{YCSB-Aにおけるスレッド数とスループットの関係}
\label{fig:a-throughput}
\end{figure}


\begin{figure}[h] 
\centering
\includegraphics[width=15cm]{ycsb-a/opposite-read-throughput}
\caption{YCSB-Aにおけるスレッド数と読み込みスループットの関係}
\label{fig:a-throughput-read}
\end{figure}


\begin{figure}[h] 
\centering
\includegraphics[width=15cm]{ycsb-a/opposite-write-throughput}
\caption{YCSB-Aにおけるスレッド数と書き込みスループットの関係}
\label{fig:a-throughput-write}
\end{figure}


YCSB-Aではスループットについては図\ref{fig:a-throughput}のようにoldに比べてsnapshotは最大61倍、latestは最大143倍のスループットとなった。また、スループットを読み込みスループット、書き込みスループットと分けてそれぞれ図\ref{fig:a-throughput-read}、図\ref{fig:a-throughput-write}で表示した。ここで、snapshotについては読み込みスループットについてはある一定以上のスレッド数では性能が低下し、また書き込みスループットについてはスレッド数の増加とともに性能が低下していることがわかる。

\begin{figure}[h] 
\centering
\includegraphics[width=15cm]{cache-a}
\caption{YCSB-Aにおけるスレッド数とキャッシュミス率の関係}
\label{fig:a-cache}
\end{figure}


このようになる原因を調べるため、スレッド数とキャッシュミス率の関係について調べ、図\ref{fig:a-cache}に示した。ここからわかるように、latestではキャッシュミス率がほとんど変化しないのに対し、snapshotではキャッシュミス率が上がっていることがわかる。§\ref{section:intro-high-gran-lock}で説明したようにsnapshotでは\textit{lookupTransform}にて同じ要素に対して二回の読み込みがある。この一回目の読み込みと二回目の読み込みの間にて\textit{setTransform}による同じ要素への書き込みが発生するとキャッシュが汚染され、二回目の読み込み時にキャッシュミスとなる。スレッド数の増加とともにこの現象が増えることがキャッシュミス率の増加に繋がり、読み込みスループットと書き込みスループットの性能低下につながったと考えられる。

\begin{figure}[h] 
\centering
\includegraphics[width=15cm]{ycsb-a/opposite-read-latency}
\caption{YCSB-Aにおけるスレッド数と読み込みレイテンシの関係}
\label{fig:a-read-latency}
\end{figure}

\begin{figure}[h] 
\centering
\includegraphics[width=15cm]{ycsb-a/opposite-read-latency2}
\caption{YCSB-Aにおけるスレッド数と読み込みレイテンシの関係 snapshotとlatestのみ}
\label{fig:a-read-latency2}
\end{figure}


\begin{figure}[h] 
\centering
\includegraphics[width=15cm]{ycsb-a/opposite-write-latency}
\caption{YCSB-Aにおけるスレッド数と書き込みレイテンシの関係}
\label{fig:a-write-latency}
\end{figure}

\begin{figure}[h] 
\centering
\includegraphics[width=15cm]{ycsb-a/opposite-write-latency2}
\caption{YCSB-Aにおけるスレッド数と書き込みレイテンシの関係 snapshotとlatestのみ}
\label{fig:a-write-latency2}
\end{figure}

読み込み・書き込み専用スレッドそれぞれのレイテンシについて図\ref{fig:a-read-latency}〜\ref{fig:a-read-latency2}に表示した。図\ref{fig:a-write-latency}のようにoldの書き込みレイテンシがスレッド数が少ない状況にて非常に悪いパフォーマンスを示しているのは、oldではジャイアントロックにより操作を逐次的にしか行えないからだと考えられる。

また、図\ref{fig:a-read-latency2}、図\ref{fig:a-write-latency2}のようにsnapshotの読み込み、書き込みレイテンシがスレッド数とともにlatestより増加しているのは、上述したようにキャッシュミスの増加によるものだと考えられる。

\begin{figure}[h] 
\centering
\includegraphics[width=15cm]{ycsb-a/opposite-abort}
\caption{YCSB-Aにおけるスレッド数とabort率の関係}
\label{fig:a-abort}
\end{figure}

スレッド数とabort率の関係について図\ref{fig:a-abort}に示した。ここからわかるように、スレッド数とabort率が線形比例していることがわかる。


\begin{figure}[h] 
\centering
\includegraphics[width=15cm]{ycsb-a/opposite-delay}
\caption{YCSB-Aにおけるスレッド数とデータの鮮度の関係}
\label{fig:a-delay}
\end{figure}

スレッド数とデータの鮮度の関係について図\ref{fig:a-delay}に示した。ここからわかるように、snapshotではスループットの低下によりスレッド数が増えるとデータの鮮度が落ちていくが、latestではスレッド数に関係なく安定して高い鮮度のデータが取得できることがわかる。


\begin{figure}[h] 
\centering
\includegraphics[width=15cm]{ycsb-a/opposite-stddiv}
\caption{YCSB-Aにおけるスレッド数とデータの同期性の関係}
\label{fig:a-stddiv}
\end{figure}

スレッド数とデータの同期性の関係について図\ref{fig:a-stddiv}に示した。ここからわかるように、latestではスレッド数に関係なくデータの同期性が安定していることがわかる。

%TODO joint=100, 60Hzでテストを行う

\section{YCSB-B}


\begin{figure}[h] 
\centering
\includegraphics[width=15cm]{ycsb-b/opposite-throughput}
\caption{YCSB-Bにおけるスレッド数とスループットの関係}
\label{fig:b-throughput}
\end{figure}


\begin{figure}[h] 
\centering
\includegraphics[width=15cm]{ycsb-b/opposite-read-throughput}
\caption{YCSB-Bにおけるスレッド数と読み込みスループットの関係}
\label{fig:b-throughput-read}
\end{figure}


\begin{figure}[h] 
\centering
\includegraphics[width=15cm]{ycsb-b/opposite-write-throughput}
\caption{YCSB-Bにおけるスレッド数と書き込みスループットの関係}
\label{fig:b-throughput-write}
\end{figure}


\begin{figure}[h] 
\centering
\includegraphics[width=15cm]{cache-b}
\caption{YCSB-Bにおけるスレッド数とキャッシュミス率の関係}
\label{fig:b-cache}
\end{figure}


スレッド数と全体のスループット、読み込みのスループット、書き込みのスループットの関係について図\ref{fig:b-throughput}〜\ref{fig:b-throughput-write}に表示した。また、スレッド数とキャッシュミス率の関係について図\ref{fig:b-cache}に表示した。ここからわかるように、書き込みが少ないために§\ref{section:ycsb-a}にて説明したようなキャッシュミス率の変化は小さくなっているため、書き込みのスループットにも大きな変化が生じなかったと考えられる。

\begin{figure}[h] 
\centering
\includegraphics[width=15cm]{ycsb-b/opposite-read-latency}
\caption{YCSB-Bにおけるスレッド数と読み込みレイテンシの関係}
\label{fig:b-read-latency}
\end{figure}

\begin{figure}[h] 
\centering
\includegraphics[width=15cm]{ycsb-b/opposite-read-latency2}
\caption{YCSB-Bにおけるスレッド数と読み込みレイテンシの関係 snapshotとlatestのみ}
\label{fig:b-read-latency2}
\end{figure}


\begin{figure}[h] 
\centering
\includegraphics[width=15cm]{ycsb-b/opposite-write-latency}
\caption{YCSB-Bにおけるスレッド数と書き込みレイテンシの関係}
\label{fig:b-write-latency}
\end{figure}

\begin{figure}[h] 
\centering
\includegraphics[width=15cm]{ycsb-b/opposite-write-latency2}
\caption{YCSB-Bにおけるスレッド数と書き込みレイテンシの関係 snapshotとlatestのみ}
\label{fig:b-write-latency2}
\end{figure}

読み込み・書き込み専用スレッドそれぞれのレイテンシについて図\ref{fig:b-read-latency}〜\ref{fig:b-read-latency2}に表示した。図\ref{fig:b-write-latency2}をみてわかるように、snapshotの方がlatestより書き込みレイテンシが優れていることがわかる。これは、latestの\textit{setTransformsXact}にて16の要素の書き込み、書き込みロックをする必要があるのに対し、snapshotの\textit{setTransform}では一つの要素の書き込み及び書き込みロックを行えばよく、さらに§\ref{section:ycsb-a}で説明したようなキャッシュミス率の変化も発生していないからだと考えられる。


\begin{figure}[h] 
\centering
\includegraphics[width=15cm]{ycsb-b/opposite-abort}
\caption{YCSB-Bにおけるスレッド数とabort率の関係}
\label{fig:b-abort}
\end{figure}

スレッド数とabort率の関係について図\ref{fig:b-abort}に示した。ここからわかるように、YCSB-BにおいてもYCSB-Aと同じようにスレッド数とabort率が線形比例していることがわかる。


\begin{figure}[h] 
\centering
\includegraphics[width=15cm]{ycsb-b/opposite-delay}
\caption{YCSB-Bにおけるスレッド数とデータの鮮度の関係}
\label{fig:b-delay}
\end{figure}

スレッド数とデータの鮮度の関係について図\ref{fig:b-delay}に示した。ここからわかるように、どの手法でもスレッド数に関係なくデータの鮮度は安定しているが、latestの方がsnapshotよりデータの鮮度が高いことがわかる。


\begin{figure}[h] 
\centering
\includegraphics[width=15cm]{ycsb-b/opposite-stddiv}
\caption{YCSB-Bにおけるスレッド数とデータの同期性の関係}
\label{fig:b-stddiv}
\end{figure}

スレッド数とデータの同期性の関係について図\ref{fig:b-stddiv}に示した。ここからわかるように、latestではYCSB-Aとは違いスレッド数の増加とともにデータの同期性が増加することがわかる。これは、スレッド数の増加とともに書き込みが増え、より最新のデータが増えるからだと考えられる。

\section{制御周期}

\begin{figure}[h] 
\centering
\includegraphics[width=17cm]{latency-frequency.pdf}
\caption{制御周期とレイテンシ}
\label{fig:latency-frequency}
\end{figure}

%TODO frequencyパラメータのところで説明は十分?

図\ref{fig:latency-frequency}は、スレッド数200の状態でfrequencyを100から100000まで変化させた時の読み込みレイテンシを表している。frequencyを上げるとレイテンシも増えることがわかる。これは、制御周期を増やすことにより並行に実行される操作が増え、スレッド間の競合が増加するからだと考えられる。

%\begin{figure}[h] 
%\centering
%\includegraphics[width=15cm]{frequency-real}
%\caption{制御周期とレイテンシ}
%\label{fig:frequency-real}
%\end{figure}

%TODO 今のままだと問題になる具体的なワークロードを示す

%oldにおいてレイテンシが問題となるワークロードを示したのが図\ref{fig:frequency-real}である。これは、thread=200、joint=1000000、read\_ratio=0.5、read\_len=10000、write\_len=10000におけるレイテンシを表している。
%
%frequencyは100なのでかなり現実的
%snapshotとlatestでは20ms、12msなのに対し、oldでは608msとなり現実的なワークロードでは問題となる。
%ここはjoint数を増やした時のレイテンシの変化を記録する?
%
%この説明は本当にいるのか?
%ヘビ型ロボットや、センサで認識した物体をTFに登録するようなワークロードでは問題となる。
%その場合には、どんなオブジェクトがどこにあるかを統一的に扱う枠組みが必要だろう。例えば今は
%でもその情報は一つのマップには集約されていない。別々のプロセスが別々のデータ型でpublishし、内部の情報はopenではない。
%例えば障害物はmove\_baseによるcostmap、信号はカメラによるYOLO、人や車の検知はLiDARのPointNetのようにデータが分散しており、今は一つにまとまってない!

\chapter{結論}
\section{結論}
既存のTFライブラリはジャイアンロックにより、アクセスが完全に逐次化されアクセスするスレッドが増えるに従ってパフォーマンスが低下する問題があった。マルチコア化が進む現在のハードウェアの性能を活かせるよう、本研究ではTFライブラリにデータベースの並行性制御法における細粒度ロッキング法を用いてこの問題を解決した。その結果、既存手法と比べて最大243倍のスループット、最大172倍高速なレイテンシとなり、細粒度ロッキング法を導入した提案手法はマルチコアの性能を活かせ、優れた応答性能をもつ事が示せた。


また、既存のTFライブラリはその仕様によって座標変換の計算時に最新のデータを参照せず、さらにジャイアントロックによりアクセスが逐次化されていることによってデータの鮮度が落ちるという問題があった。本研究ではデータベースの並行性制御法における2PLを適用することにより、複数の座標変換の最新のデータをatomicに取得できるインターフェイス(\textit{lookupLatestTransformXact})、及び複数の座標変換の最新のデータをatomicに更新するインターフェイス(\textit{setTransformsXact})を提供することによって解決した。これにより、複数のデータの最新の座標変換情報の読み込み・書き込みを効率的にatomicに行えるようになった。その結果、既存手法と比べて最大132倍のデータ鮮度となり、2PLを導入した提案手法はデータの鮮度を向上させる事ができる事を示した。また、既存手法と比べ最大257倍のスループット、最大282倍高速化したレイテンシとなり、このインターフェイスはスループットとレイテンシにおいても既存手法より優れている事を示した。

\section{今後の課題}
本研究のようにロボットにおいてデータベースの並行性制御法を導入する研究アプローチは他になく、ロボットの高性能化のためには並行性制御法の導入が必要である事を示した。

本研究ではTFライブラリの木構造は変化しないと仮定したが、実際には時間とともにフレームが追加・削除され、またフレーム間の親子関係が変わることがある。データベースの並行性制御法においてはこのような要素の追加・削除はIDMモデルで扱われ、高度な並行性制御にはphantom anomalyを避ける必要がある。要素の追加・削除についても扱うことが今後の課題である。

% アプリケーションはどこに書こうか
本研究のアプリケーションは図\ref{fig:snake}で取り上げたようなヘビ型ロボット以外にも、自動運転車におけるエッジクラウドでの応用が挙げられる。自動運転車におけるエッジクラウドでは、一部のローカルな区域における自動運転車両や発生した障害物の位置関係を管理することによって、渋滞の緩和や障害物の回避を行う事ができる。リアルタイム性が求められ、大量の自動運転車両や障害物の位置関係をTFライブラリで管理するには、本研究で提案した手法を用いる事が有用である。
また、以下のような問題にも対処する必要がある。
\begin{itemize}     
	\item 区域内への自動運転車両の出入りや障害物の登録・削除によって大量のフレームの追加・削除が発生するため、上述したようなphantom anomalyの回避が必要になる
	\item TFライブラリでは過去一定期間の座標変換情報を管理しているため、大量のフレームの追加・削除が発生する場合には適切なGCも必要になる
	\item エッジクラウドの電源がロストした場合にすぐに復旧し、いち早くサービスを提供できるようにCrash Recoveryを実装する必要もある
\end{itemize}

こういったアプリケーションにおいては、もはやデータベースの技術を応用することは避けられないだろう。

本研究では並行性制御の具体的なアルゴリズムとしては2PLを実装したが、Siloを実装した場合の性能比較も行うことが今後の課題である。

また、よりロボット向けのワークロードに対応するため、TFに対する操作に優先順位をつけ、優先順位が高い操作をなるべく早く終わらせるために優先順位キューを実装することも検討が必要である。

ロボットに並行性制御法を導入する対象として本研究ではTFライブラリを取り上げたが、他にもROSで頻繁に使用されるmove\_baseなどのパッケージにおいても並行性制御法の導入の検討が必要である。

\chapter*{謝辞}
\addcontentsline{toc}{chapter}{\numberline{}謝辞}

本研究を進めるにあたり、慶應義塾大学准教授川島英之先生、筑波大学システム情報系情報工学域大矢晃久、筑波大学システム情報系情報工学域萬礼応先生に頂きました優れた御指導により、私の研究はとても有意義で満ち足りたものとなりました。また、慶應義塾大学川島研究会秘書藤川綾様には幾多の手続きを丁寧にサポートして頂き,円滑な出張や書類作成,研究環境整備を行うことができました。この研究に関わっていただいたすべての方に深く感謝を申し上げます。

% 参考文献(References)
\newpage
\addcontentsline{toc}{chapter}{\numberline{}参考文献}
\renewcommand{\bibname}{参考文献}


%% 参考文献に bibtex を使う場合
%\bibliographystyle{junsrt}
%\bibliography{ref}

%% 参考文献を直接ファイルに含めて書く場合
	
\begin{thebibliography}{99}

\bibitem{report1} "自律移動型ロボットの世界市場は2027年まで年平均成長率19.6%で成長すると予想される", \url{https://www.jiji.com/jc/article?k=000004775.000067400&g=prt}

\bibitem{report2} "IDC Japan/自律移動型ロボットが2023年に561億円の市場規模に", \url{https://www.lnews.jp/2019/05/l0514309.html}

\bibitem{tf} T. Foote, "tf: The transform library," 2013 IEEE Conference on Technologies for Practical Robot Applications (TePRA), 2013, pp. 1-6, doi: 10.1109/TePRA.2013.6556373.

\bibitem{ros} M. Quigley, K. Conley, B. P. Gerkey, J. Faust, T. Foote, J. Leibs, R. Wheeler, and A. Y. Ng, “Ros: an open-source robot operating system,” in ICRA Workshop on Open Source Software, 2009.

\bibitem{ssm} 竹内栄二朗(2008). 移動ロボットの基本機能のモジュール化と環境地図生成に関する研究, 筑波大学博士 (工学) 学位論文.

\bibitem{2PL} Philip A. Bernstein and Nathan Goodman, "Concurrency Control in Distributed Database Systems" in ACM Computing Surveys, 1981, pp. 185-221

\bibitem{MOCC} Tianzheng Wang and Hideaki Kimura. Mostly-optimistic concurrency control for high- lycontended dynamic workloads on a thousand cores. Proceedings of the VLDB En- dowment, Vol. 10, No. 2, p. p. 49âĂŞ60, 2016.

\bibitem{silo} Stephen Tu, Wenting Zheng, Eddie Kohler †, Barbara Liskov,
and Samuel Madden. Speedy transactions in multicore in-memory
databases. In SOSP, pages 18–32. ACM, 2013

\bibitem{Cicada} Hyeontaek Lim, Michael Kaminsky, and David G Andersen. Cicada: Dependably fast multi-core in-memory transactions. InProceedings of the 2017 ACM International Conference onManagement of Data, p. p. 21âĂŞ35, 2017.

\bibitem{ccbench} Takayuki Tanabe, Takashi Hoshino, Hideyuki Kawashima, Osamu Tatebe: An Analysis of Concurrency Control Protocols for In-Memory Databases with CCBench. Proc. VLDB Endowment 13(13): 3531-3544 (2020).

\bibitem{Aqua} 南祐衣, 鈴木汰一, 山倉拓海, 構勇海, 潘鴻遠, 及川裕介 (筑波大), 荻原湧志, 川島英之 (慶應大), 伊達央, 萬礼応 (筑波大): つくばチャレンジ 2021 における 筑波大学知能ロボット研究室チーム Aqua の取り組み. 第22回 計測自動制御学会.

\bibitem{snake-sim} 伊達央, 阿部太郎: ヘビ型ロボットのオドメトリと遠隔操作支援. 第8回横幹連合コンファレンス.

\bibitem{buffer-core} "BufferCore.h", \url{https://github.com/ros/geometry2/blob/melodic-devel/tf2/include/tf2/buffer_core.h}

\bibitem{lookupTransform} "BufferCore.cpp lookupTransform", \url{https://github.com/ros/geometry2/blob/melodic-devel/tf2/include/tf2/buffer_core.h}

\bibitem{gaia} "GAIA platform", \url{https://www.gaiaplatform.io}

\bibitem{ros2} "ROS2", \url{https://docs.ros.org/en/rolling/}

\bibitem{autoware} "Autoware", \url{https://tier4.jp/en/autoware/}

\bibitem{ndt_matching} "ndt\_matching", \url{https://github.com/Autoware-AI/core_perception/tree/master/lidar_localizer/nodes/ndt_matching}

\bibitem{ycsb} Brian F Cooper, Adam Silberstein, Erwin Tam, Raghu Ramakrishnan, and Russell Sears. Benchmarking cloud serving systems with ycsb. InProceedings of the 1st ACM symposiumon Cloud computing, p. p. 143âĂŞ154, 2010.

\bibitem{nowait} Guna Prasaad, Alvin Cheung and Dan Suciu. Improving High Contention OLTP Performance via Transaction Scheduling. 

\bibitem{ros-geometry2} "ros/geometry github repository", \url{https://github.com/ros/geometry2} 

\bibitem{ros2-geometry2} "ros2/geometry2 github repository", \url{https://github.com/ros2/geometry2}

\bibitem{ogiwara-geometry2} "Ogiwara-CostlierRain464/geometry2 github repository", \url{https://github.com/Ogiwara-CostlierRain464/geometry2}

\bibitem{Teseo} Dean De Leo and Peter Boncz. Teseo and the Analysis of Structural Dynamic Graphs. Proc. VLDB Endowment 13(13): 3531-3544 (2020).

\end{thebibliography}


\end{document}
